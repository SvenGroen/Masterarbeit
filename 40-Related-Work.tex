\chapter{Related Work}
\label{ch:relatedWork}


%-------------------------------------------------------------------------
\section{Generative Adversarial Networks Models}
\label{ch:relatedWork-generativeAdversarialNetworksModels}


mirza2014ConditionalGenerativeAdversarial
% what improvements have been done to GAN's
- gans struggled with generating discrete variables \cite{torfi2020CorGANCorrelationCapturingConvolutionala}
- State of the art Gan approaches only focused on continouse and categorical types, overlooking mixed data types \cite{zhao2022CTABGANEnhancingTabular}

- conditional GAN \cite{mirza2014ConditionalGenerativeAdversarial}


% mathematical formulation
% Advantages / Problems / Challenges (Mode Collapse, etc.)
- remarkable performance generating syntehtic images and time series data \cite{mckeever2020SynthesisingTabularDatasets}
- struggle with mode collapse --> generate same sample \cite{torfi2020CorGANCorrelationCapturingConvolutionala}
- wasserstein loss helps against mode collapse \cite{frogner2015LearningWassersteinLoss} \cite{arjovsky2017WassersteinGenerativeAdversarial} and has been implemented in many gans (e.g. \cite{zhao2022CTABGANEnhancingTabular})
- gradient penalty \cite{gulrajani2017ImprovedTrainingWasserstein}


\subsection{CTGAN}
\label{ch:relatedWork-generativeAdversarialNetworksModels-ctgan}

\subsection{CTAB GAN}
\label{ch:relatedWork-generativeAdversarialNetworksModels-ctabGAN}
%-------------------------------------------------------------------------
\section{Transformer Models}
\label{ch:relatedWork-transformers}

TabTransformer only embeddeds categorical values --> cont not through attention --> no correlation capture \cite{somepalli2021SAINTImprovedNeural}

- TabNet is one of the first transformer-based models for tabular data \cite{borisov2022DeepNeuralNetworks}
- padhi2021TabularTransformersModeling


%-------------------------------------------------------------------------
\section{Diffusion Models}
\label{ch:relatedWork-diffusionModels}


\subsection{Diffusion Probabilistic Models for Tabular Data}
\label{ch:preliminaries-generativeAlgorithms-diffusionProbabilisticModelsTabularData}

% How can Diffusion Probabilistic Models be used for tabular data
% Challenges of using Diffusion Probabilistic Models for tabular data (mixed data types --> different noising process, etc.)

\subsection{Tab-DDPM}
\label{ch:relatedWork-diffusionModels-tabDDPM}

% Paper for image synthesis
% paper for diffusion for missing data entries
% Paper for tabular data synthesis


%-------------------------------------------------------------------------
