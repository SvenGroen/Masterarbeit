\documentclass[11pt,DIV=15,BCOR=20mm]{scrbook}

% Import von Paketen und Optionen die das gesamte Dokument betreffen
% sind in myPreamble.sty ausgelagert.
\usepackage{myPreamble}
% Glossar Einträge:

\newglossaryentry{model}{
    name={Model},
    description={A model in machine learning or deep learning is a mathematical representation that captures the relationship between inputs and outputs in data.
    It is used to make predictions about new, unseen data by applying mathematical operations to the inputs.
    Deep learning models are a subset of parametric models in which the parameters are represented as weights in a neural network and are changed during training to reduce the discrepancy between expected and actual outcomes.
    Machine learning models include but are not limited to decision trees, logistic regression, random forests, and linear regression.
    Convolutional neural networks (CNNs), recurrent neural networks (RNNs), and autoencoders are a few examples of deep learning models \cite{parsons2021WhatMachineLearning}},
    plural={models}
}

\newglossaryentry{lvm}{
    name={Latent variable model},
    description={
        Latent variable models are a type of probabilistic model that use latent or hidden variables to explain the observed data. 
        They model the relationship between the observed data and the latent variables through a set of conditional probability distributions.
        The goal is often to use this joint distribution to make predictions or perform inferences about the data, which typically involves marginalizing over the latent variables. 
        This means integrating over all possible values of the latent variables to obtain the probability distribution of the observed data \cite{skrondal2007LatentVariableModelling, bishop1998latent}
    },
    plural={latent variable models}
}

\newglossaryentry{mc}{
    name={Markov chain},
    description={A Markov chain, also known as a Markov process, 
    represents a probabilistic model where the likelihood of an event occurring is determined 
    exclusively by the outcome of the preceding event in a series of potential occurrences \cite{gagniuc2017MarkovChainsTheory}
    },
    plural={Markov chains}
}

\newglossaryentry{oh}{
    name={One-Hot},
    description={A One-Hot vector is a binary vector used to represent categorical data in machine learning and deep learning algorithms.
    It is characterized by having exactly one element set to 1 (the "hot" element) and all other elements set to 0. 
    Suppose a dataset contains a "color" feature with three distinct categories: red, green, and blue. 
    Using one-hot encoding, a unique one-hot vector for each color would look like the following:\\
    Red: [1, 0, 0]\\
    Green: [0, 1, 0]\\
    Blue: [0, 0, 1]
    },
    plural={One-Hot}
}

\newglossaryentry{pd}{
    name={pandas},
    description={
        Pandas \cite{mckinney-proc-scipy-2010} is an open-source Python library that provides high-performance, easy-to-use data structures, and data analysis tools. 
        It is designed to facilitate data manipulation, cleaning, analysis, and visualization in Python
    },
    plural={pandas}
}

\newglossaryentry{gdpr}{
    name={GDPR}, 
    description={The General Data Protection Regulation (GDPR) \cite{european_commission_regulation_2016} is a data privacy and security law passed by the European Union, which imposes standards and potential penalties on organizations worldwide that handle data related to people within the European Union
    },
    % first={General Data Protection Regulation (GDPR)\glsadd{gdprg}}, 
    % plural={GDPR}
}

\newglossaryentry{gdpraccr}{
    type=\acronymtype, 
    name={GDPR}, 
    description={General Data Protection Regulation},
    first={General Data Protection Regulation (GDPR)\glsadd{gdpr}}, 
    see=[Glossary:]{gdpr}
}


\newglossaryentry{toml}{
    name={TOML}, 
    description={TOML, which stands for Tom's Obvious, Minimal Language \cite{preston-werner2021TOMLTomObviousa}, is an open-source configuration file format that aims to be easy to read and write. 
    TOML documents contain sections, indicated with square brackets which in turn contain key-value pairs, specifying the overall configuration},
    % first={General Data Protection Regulation (GDPR)\glsadd{gdprg}}, 
    % plural={GDPR}
}

\newglossaryentry{tomlaccr}{
    type=\acronymtype, 
    name={TOML}, 
    description={Tom's Obvious, Minimal Language},
    first={TOML (short for "Tom's Obvious, Minimal Language")\glsadd{toml}}, 
    see=[Glossary:]{toml}
}


% Akkronyme
\newacronym{smote}{SMOTE}{Synthetic Minority Oversampling Technique}
\newacronym{vgm}{VGM}{Variational Gaussian Mixture model}
\newacronym{cnn}{CNN}{Convolutional Neural Network}
\newacronym{rnn}{RNN}{Recurrent Neural Network}
\newacronym{lstm}{LSTM}{Long Short-Term Memory}
\newacronym{gru}{GRU}{Gated Recurrent Unit}
% \newacronym{gdpr}{GDPR}{General Data Protection Regulation}
\newacronym{vae}{VAE}{Variational Autoencoders}
\newacronym{pca}{PCA}{Principal Component Analysis}
\newacronym{kl}{KL}{Kullback-Leibler}
\newacronym{gan}{GAN}{Generative Adversarial Network}
\newacronym{mlp}{MLP}{Multilayer Perceptron}
\newacronym{relu}{ReLU}{Rectified Linear Activation Unit}
\newacronym{ddpm}{DDPM}{Denoising Diffusion Probabilistic Model}
\newacronym{cdf}{CDF}{Cumulative Density Function}
\newacronym{mse}{MSE}{Mean-Squared Error}
\newacronym{pcd}{PCD}{Pairwise Correlation Difference}
\newacronym{dp}{DP}{Differential Privacy}
\newacronym{pmse}{pMSE}{propensity Mean Squared Error}
\newacronym{rmse}{RMSE}{Root Mean Squared Error}
\newacronym{gmm}{GMM}{Gaussian Mixture Model}
\newacronym{nlp}{NLP}{Natural Language Processing}
\newacronym{gpu}{GPU}{Graphics Processing Unit}
\newacronym{bgm}{BGM}{Bayesian Gaussian Mixture Model}
\newacronym{dcr}{DCR}{Distance to Closest Record}
\newacronym{cpu}{CPU}{Central Processing Unit}
\newacronym{ft}{FT}{Feature Tokenization}
\newacronym{rq}{RQ}{Research Question}
\makeglossaries

\begin{document}
% --- TITELSEITE ---
\begin{titlepage}

	% Fehler "destination with the same identifier" unterdrücken...
  \setcounter{page}{-1}

	% Titelseite
	\begin{figure}[h]
		\begin{minipage}[b]{62mm}
			\includegraphics[width=62mm]{images/unilogo}
		\end{minipage}
		\hspace{4cm}
		%\begin{minipage}[b]{59mm}
		%	\includegraphics[width=59mm]{images/minlogo}
		%\end{minipage}
	\end{figure}

	\vfill
	
	\begin{center}
		% Diplomarbeit 
		\noindent { \huge
			Masterarbeit \\
		}
		\vspace{14mm}
		% Titel
		\noindent \textbf{\huge
		  Vorlage für Abschlussarbeiten
		}
		\vspace{60mm}	
	\end{center}
	
	\vfill
	
	\noindent \textbf{Max Peter Mustermann} \\
	\noindent \rule{\textwidth}{0.4mm} 
	\noindent{\textrm{Sven.Groen@studium.uni-hamburg.de}} \\
	\noindent{\textrm{Studiengang IT Management und Consulting}} \\
	\noindent{\textrm{Matr.-Nr. 12345678}} \\
	\begin{tabbing}
	\hspace{8em} \=  \kill
	Erstgutachter: \> Professor A. Ersthelfer \\
	Zweitgutachter: \> Professor Z. Eswirdschonwerden \\
	~ \\
	Abgabe: 03.2021
	\end{tabbing}

\end{titlepage}



% VERZEICHNISSE (Inhaltsverzeichnis, Abkürzungen)
% Vorspann einleiten --> Seitennummerierung römisch
\frontmatter

% --- Inhaltsverzeichnis --- 

\renewcommand{\contentsname}%
  {Table of Contents}%

\tableofcontents
\cleardoublepage

% % Glossar Einträge:

\newglossaryentry{model}{
    name={Model},
    description={A model in machine learning or deep learning is a mathematical representation that captures the relationship between inputs and outputs in data.
    It is used to make predictions about new, unseen data by applying mathematical operations to the inputs.
    Deep learning models are a subset of parametric models in which the parameters are represented as weights in a neural network and are changed during training to reduce the discrepancy between expected and actual outcomes.
    Machine learning models include but are not limited to decision trees, logistic regression, random forests, and linear regression.
    Convolutional neural networks (CNNs), recurrent neural networks (RNNs), and autoencoders are a few examples of deep learning models \cite{parsons2021WhatMachineLearning}},
    plural={models}
}

\newglossaryentry{lvm}{
    name={Latent variable model},
    description={
        Latent variable models are a type of probabilistic model that use latent or hidden variables to explain the observed data. 
        They model the relationship between the observed data and the latent variables through a set of conditional probability distributions.
        The goal is often to use this joint distribution to make predictions or perform inferences about the data, which typically involves marginalizing over the latent variables. 
        This means integrating over all possible values of the latent variables to obtain the probability distribution of the observed data \cite{skrondal2007LatentVariableModelling, bishop1998latent}
    },
    plural={latent variable models}
}

\newglossaryentry{mc}{
    name={Markov chain},
    description={A Markov chain, also known as a Markov process, 
    represents a probabilistic model where the likelihood of an event occurring is determined 
    exclusively by the outcome of the preceding event in a series of potential occurrences \cite{gagniuc2017MarkovChainsTheory}
    },
    plural={Markov chains}
}

\newglossaryentry{oh}{
    name={One-Hot},
    description={A One-Hot vector is a binary vector used to represent categorical data in machine learning and deep learning algorithms.
    It is characterized by having exactly one element set to 1 (the "hot" element) and all other elements set to 0. 
    Suppose a dataset contains a "color" feature with three distinct categories: red, green, and blue. 
    Using one-hot encoding, a unique one-hot vector for each color would look like the following:\\
    Red: [1, 0, 0]\\
    Green: [0, 1, 0]\\
    Blue: [0, 0, 1]
    },
    plural={One-Hot}
}

\newglossaryentry{pd}{
    name={pandas},
    description={
        Pandas \cite{mckinney-proc-scipy-2010} is an open-source Python library that provides high-performance, easy-to-use data structures, and data analysis tools. 
        It is designed to facilitate data manipulation, cleaning, analysis, and visualization in Python
    },
    plural={pandas}
}

\newglossaryentry{gdpr}{
    name={GDPR}, 
    description={The General Data Protection Regulation (GDPR) \cite{european_commission_regulation_2016} is a data privacy and security law passed by the European Union, which imposes standards and potential penalties on organizations worldwide that handle data related to people within the European Union
    },
    % first={General Data Protection Regulation (GDPR)\glsadd{gdprg}}, 
    % plural={GDPR}
}

\newglossaryentry{gdpraccr}{
    type=\acronymtype, 
    name={GDPR}, 
    description={General Data Protection Regulation},
    first={General Data Protection Regulation (GDPR)\glsadd{gdpr}}, 
    see=[Glossary:]{gdpr}
}


\newglossaryentry{toml}{
    name={TOML}, 
    description={TOML, which stands for Tom's Obvious, Minimal Language \cite{preston-werner2021TOMLTomObviousa}, is an open-source configuration file format that aims to be easy to read and write. 
    TOML documents contain sections, indicated with square brackets which in turn contain key-value pairs, specifying the overall configuration},
    % first={General Data Protection Regulation (GDPR)\glsadd{gdprg}}, 
    % plural={GDPR}
}

\newglossaryentry{tomlaccr}{
    type=\acronymtype, 
    name={TOML}, 
    description={Tom's Obvious, Minimal Language},
    first={TOML (short for "Tom's Obvious, Minimal Language")\glsadd{toml}}, 
    see=[Glossary:]{toml}
}


% Akkronyme
\newacronym{smote}{SMOTE}{Synthetic Minority Oversampling Technique}
\newacronym{vgm}{VGM}{Variational Gaussian Mixture model}
\newacronym{cnn}{CNN}{Convolutional Neural Network}
\newacronym{rnn}{RNN}{Recurrent Neural Network}
\newacronym{lstm}{LSTM}{Long Short-Term Memory}
\newacronym{gru}{GRU}{Gated Recurrent Unit}
% \newacronym{gdpr}{GDPR}{General Data Protection Regulation}
\newacronym{vae}{VAE}{Variational Autoencoders}
\newacronym{pca}{PCA}{Principal Component Analysis}
\newacronym{kl}{KL}{Kullback-Leibler}
\newacronym{gan}{GAN}{Generative Adversarial Network}
\newacronym{mlp}{MLP}{Multilayer Perceptron}
\newacronym{relu}{ReLU}{Rectified Linear Activation Unit}
\newacronym{ddpm}{DDPM}{Denoising Diffusion Probabilistic Model}
\newacronym{cdf}{CDF}{Cumulative Density Function}
\newacronym{mse}{MSE}{Mean-Squared Error}
\newacronym{pcd}{PCD}{Pairwise Correlation Difference}
\newacronym{dp}{DP}{Differential Privacy}
\newacronym{pmse}{pMSE}{propensity Mean Squared Error}
\newacronym{rmse}{RMSE}{Root Mean Squared Error}
\newacronym{gmm}{GMM}{Gaussian Mixture Model}
\newacronym{nlp}{NLP}{Natural Language Processing}
\newacronym{gpu}{GPU}{Graphics Processing Unit}
\newacronym{bgm}{BGM}{Bayesian Gaussian Mixture Model}
\newacronym{dcr}{DCR}{Distance to Closest Record}
\newacronym{cpu}{CPU}{Central Processing Unit}
\newacronym{ft}{FT}{Feature Tokenization}
\newacronym{rq}{RQ}{Research Question}
%\printglossaries[type=\acronymtype, title=Abkurzungsverzeichnis]
%\abbreviationsname[Abkürzungsverzeichnis]
\printglossary[type=main, title={Glossary}]
\printglossary[type=acronym, title={List of Abbreviations}]

%\printglossaries

%\let\cleardoublepage\clearpage % löscht die leere seite
\cleardoublepage
\phantomsection
% --- Figurenverzeichnis und Tabellenverzeichnis ---
\addcontentsline{toc}{chapter}{\listfigurename} % damit es in Inhaltsverzeichnis kommt
\listoffigures
%\let\cleardoublepage\clearpage % löscht die leere seite
\cleardoublepage
\phantomsection
\addcontentsline{toc}{chapter}{\listtablename}% damit es in Inhaltsverzeichnis kommt
\listoftables
\cleardoublepage
%\let\cleardoublepage\clearpage

% Hauptteil einleiten --> Seitennummerierung wieder arabisch
\mainmatter

% --- Hauptteil ---
% \phantomsection
% \addcontentsline{toc}{chapter}{Abstract}
\thispagestyle{empty}
\begin{center}
	\textbf{\LARGE Abstract}
\end{center}
	In this paper, we present a new approach for solving a long-standing problem in the field of X.
\cleardoublepage

\chapter{Acknowledgements}
\label{ch:Acknowledgements}
\chapter{Introduction}
\label{ch:introduction}

\section{Problem Statement and Motivation} 
\label{ch:intro-problemStatement}

Tabular data is an essential tool in our society due to its ability to organize and present data in a structured manner that can be easily analyzed and interpreted by humans and machines alike. 
This makes it an indispensable tool for decision-making processes in many fields, including finance, medical research, and many more.
In the era of deep learning and big data, the need for accurate and reliable data in high quantities is paramount.
For machine and deep learning models, training data is essential to perform any kind of desired inference.
Gathering and accumulating real world data is still expensive and restricted through regulations like GDPR.
In recent years, generative modeling approaches address these issues by creating a synthetic copy of the original tabular dataset.
This synthetic data version should maintain the important statistical properties and correlations while not disclosing any information from the real original dataset \cite{goodfellow2020GenerativeAdversarialNetworks, mottini2018AirlinePassengerName}.
While in the image domain, generative models already show human-like capability to produce highly realistic images \cite{dhariwal2021DiffusionModelsBeat},
tabular data synthesis approaches still show room for improvements \cite{chundawat2022UniversalMetricRobust}.
One of the reasons for this, is heterogeneous nature of tabular data, which generative models need to reproduce.
Tabular data usually consist of a mixture of numerical and categorical distributions with dependencies, which is difficult to capture and to reproduce \cite{borisov2022DeepNeuralNetworks}.
So far, \gls{gan} based approaches have shown state-of-the-art capabilities to create synthetic tabular data, which have also been the best model for synthetic image generation.
However, diffusion based models have recently been able to outperform \glspl{gan} on the task of image synthesis \cite{dhariwal2021DiffusionModelsBeat}.
As a result, researchers have tried to use diffusion models to generate synthetic tabular data with promising results \cite{kotelnikov2022TabDDPMModellingTabular, zheng2022DiffusionModelsMissing}.

\cite{dhariwal2021DiffusionModelsBeat} argued, that \glspl{gan} have shown superior performance against diffusion due to the fact, that they have been investigated more deeply by researchers.
The authors were able to show in their work, that with a few improvements, diffusion models were capable of outperforming \glspls{gan} on image synthesis.
A similar observation can be made in the tabular data domain, where the majority of recent publications propose \gls{gan} based solutions.
First results \cite{kotelnikov2022TabDDPMModellingTabular} indicate the superior performance of diffusion models over \glspl{gan} extends to the tabular data synthesis domain.
Nevertheless, applying diffusion models to tabular data is still a novel approach and further research is required \cite{borisov2022DeepNeuralNetworks}.

 
\section{Goals and Research Questions}
\label{ch:intro-goals}
The overall goal of this thesis is to further investigate how diffusion models can be successfully applied to the task of tabular data synthesis.
This work will build upon the work of \cite{kotelnikov2022TabDDPMModellingTabular} and will explore, how the existing approach, called TabDDPM, can be further improved by adapting 
already existing tabular processing mechanism from previous tabular data synthesis solutions.
Furthermore, the evaluation and comparison of the diffusion model to various other generative modeling techniques will be extended through the usage and analysis
of multiple similarity metrics as proposed by \cite{chundawat2022UniversalMetricRobust}.
The goal of this extended evaluation is to uncover possible unknown modeling properties of diffusion models that have not been discovered in previous works.

To summarize, this thesis will investigate the following \glspl{rq}:

\begin{description}
    \item[\gls{rq}1:] What are the effects on the generative capability of the tabular data synthesis model TabDDPM by adding additional tabular data processing mechanisms to the existing generation pipeline.
    \item[\gls{rq}2:] How does TabDDPM (and its variants, produced to answer \gls{rq}) compare to other generative models in terms of several similarity metrics, including \cite{chundawat2022UniversalMetricRobust}.
\end{description}

Answering the above questions contributes to the overall research domain by showing possible strengths and weaknesses of diffusion models
as a new approach in the tabular data synthesis domain. 

\section{Proceeding}
\label{ch:intro-proceeding}
In order to answer the above stated \glspl{rq}, the following steps have been taken.
Prior to any experiments, a literature search has been performed about possible different approaches to process tabular data and tabular data synthesis and how to evaluate it.
From this research, a selection of tabular processing mechanism was made based on predefined criteria (\autoref{ch:Concept-criteria}).
Using the existing TabDDPM model \cite{kotelnikov2022TabDDPMModellingTabular}, the results of the original authors needed to be replicated and validated in a first step.
Afterwards, the software was extended to integrate the selected tabular processing mechanism.
The already existing evaluation proposed in \cite{kotelnikov2022TabDDPMModellingTabular} was extended by an additional, more elaborate evaluation \cite{akim2023TabDDPMModellingTabular}.
Lastly, the different model versions have been trained and the results have been compared and analyzed.

\section{Contributions}
\label{ch:intro-contributions}

\section{Outline}
\label{ch:intro-outline}

\chapter{Preliminaries}
\label{ch:preliminaries}

%-------------------------------------------------------------------------
\section{Data Synthesis}
\label{ch:preliminaries-dataSynthesis}

\subsection{Synthetic Data}
\label{ch:preliminaries-dataSynthesis-syntheticData}

% Definition of synthetic data
% Types of synthetic data (Audio, Image, Text, Tabular)
% usage of synthetic data
- synthetic data can be use for testing and validation of applications \cite{gilad2021SynthesizingLinkedData} and machine learning models \cite{dahmen2019SynSysSyntheticData}
- allows data sharing \cite{hernandez2022SyntheticDataGeneration} while fulfilling regulatory and privacy constraints \cite{zhao2022CTABGANEnhancingTabular} (complies with GDPR \cite{zhao2022CTABGANEnhancingTabular})
- rebalance datasets with synthetic data if dataset is skewed \cite{zhao2022CTABGANEnhancingTabular}
- access to data still major bottleneck for researches of ml/dl-models \cite{fan2020RelationalDataSynthesisa}
- access is often restricted due to sensitivity of data (medical records) \cite{esteban2017RealvaluedMedicalTimea}
- can be used as additional training data \cite{kim2021OCTGANNeuralODEbased}

- can be used to create test data for software applications because developers might:\cite{whiting2008CreatingRealisticScenariobased}
    - real data may not be available or not visible for certain developers for security reasons \cite{whiting2008CreatingRealisticScenariobased}
    - test data needs to fulfil certain requirements depending on the testing scenario and obtaining the needed characteristics from real data is time-consuming \cite{whiting2008CreatingRealisticScenariobased}


% differences between real and synthetic data
- synthetic data is cheap to generate \cite{leminh2021AirGenGANbasedSynthetica}
- and can be combined with real data for training purposes \cite{leminh2021AirGenGANbasedSynthetica}

% Challenges of synthetic data (privacy, scalability, availability, etc.)

- two competing objectives in generating synthetic data: \cite{little2021GenerativeAdversarialNetworksa}
    1. high data utility \cite{little2021GenerativeAdversarialNetworksa}
    2. low disclosure risk \cite{little2021GenerativeAdversarialNetworksa}


\subsection{Tabular Data}
\label{ch:preliminaries-dataSynthesis-tabularData}

Tabular data is the most one of the most common forms of structured structured data \cite{hernandez2022SyntheticDataGeneration} used to store, classify and share information \cite{pilaluisa2022ContextualWordEmbeddings}.
A table is made up of individual cell entries stored in rows and columns where rows can be seen as individual data points and columns are the different features \cite{borisov2022DeepNeuralNetworks, yoon2020VIMEExtendingSuccess}
This format is the most common way to maintain massive databases \cite{esmaeilpour2022BidiscriminatorGANTabular, yoon2020VIMEExtendingSuccess} 
and is crucial for applications that store heterogeneous information such as demographics, medical or financial information \cite{borisov2022DeepNeuralNetworks, yoon2020VIMEExtendingSuccess}.
Tabular data consists of multiple attribute types, such as categorical or continuous data types\cite{borisov2022DeepNeuralNetworks}.
While continuous values are of quantitative nature and stored in a numerical format, categorical values are made up by one value out of a limited set of values \cite{lederrey2022DATGANIntegratingExperta, lane2003IntroductionStatistics}.
Categorical attribute types can be further subdivided into binary, only two possible values, nominal, at least three possible values that do not follow any order and lastly ordinal with at least three entries where the values follow have an underlying order \cite{lederrey2022DATGANIntegratingExperta}.
This work adapts the formal definition of a table from \cite{xu2019ModelingTabularData}:

\begin{displayquote}
A table $T$ contains $N_{con}$ continuous columns and $N_{cat}$ categorical columns. [TODO: Definition ohne Discrete values]
\end{displayquote}

It is also possible that tabular data can contain other special data types like dates or timestamps which often contain information about the specific time a datapoint was recorded \cite{hernandez2022SyntheticDataGeneration}.
This kind of tabular data can be considered as dynamic tabular data, where individual records, \ie rows, can be dependent on each other, also known as a multivariate time series \cite{padhi2021TabularTransformersModeling}.
In static tabular data on the other hand the individual rows are independent from each other \cite{padhi2021TabularTransformersModeling}.
Hence, the order of rows and columns does not carry any meaning \cite{somepalli2021SAINTImprovedNeural}.
While the order of rows and columns does not matter, the individual values in one cell may vary well depend on values of another cell \cite{lederrey2022DATGANIntegratingExperta}.
An example for such an interdependency could occur in a demographics table, where an individual's legal status may depend on their age, 
for instance, a person under 18 years old is considered a minor and has different legal rights and responsibilities compared to an adult.

The authors of \cite{borisov2022DeepNeuralNetworks} identify four possible challenges when working with tabular data in a learning context.
The first identified challenge is the low quality of the data. Typical quality issues include missing values, noise in the data, extreme data points, data inconsistencies, class imbalance or high dimensionalities after preprocessing \cite{borisov2022DeepNeuralNetworks}[CONFIRM 1].
Secondly the missing irregular spatial dependencies of tabular data. Other common data formats like images or audio are homogeneous, 
meaning that they consists of only 1 feature type \cite{borisov2022DeepNeuralNetworks}.
Since tabular data consists of multiple features, made up of a mixture of categorical and numerical values, it is a heterogeneous data format with data points as rows and features as columns \cite{borisov2022DeepNeuralNetworks}.
This makes is especially challenging for neural networks to work with since the correlations between the features is weaker because they often do not have any form of spatial or semantic relationship like image or text data has \cite{borisov2022DeepNeuralNetworks, yoon2020VIMEExtendingSuccess}.
The third challenge is about the dependency on preprocessing. \cite{borisov2022DeepNeuralNetworks} highlights the importance of a preprocessing and explicit feature construction step that is necessary when working with tabular data in a deep learning context.
This preprocessing step is crucial since it does not only strongly influences the performance of deep learning models \cite{gorishniy2022EmbeddingsNumericalFeatures} 
it also introducing new challenges. Depending on the preprocessing strategy (\autoref{sec:preprocessing}) it is possible to create a very sparse feature matrix, create a synthetic ordinal ordering of a nominal variable or lose some information during the conversion of the data \cite{borisov2022DeepNeuralNetworks}.
The last and fourth challenge concerns the importance of a single feature. In homogeneous data multiple features need to change in order to change the class of the data. 
For heterogeneous tabular data a small change in one feature variable can already alter the class of the row. 
\cite{borisov2022DeepNeuralNetworks} illustrates this with the example of an image, where multiple pixels (\ie features) need to change in a coordinated manner in order to change the content (\ie class) of the image.
For tabular data a single change in a cell can change the prediction of a predictive model \cite{borisov2022DeepNeuralNetworks}. 
Consider a model that has to predict whether an individuals income is higher or lower than US\$ 50.000 per year \cite{Dua:2019} based on demographic information of that individual.
Switching an individuals "education" value from "Preschool" to "Doctorate" would likely cause the prediction to change from "<=50K" to ">50K".




%intro
%tabular data is one of the most common ways to maintain massive databases \cite{esmaeilpour2022BidiscriminatorGANTabular} \cite{yoon2020VIMEExtendingSuccess}
%tabular data is most common form of structured data \cite{hernandez2022SyntheticDataGeneration}
%tabular data format is used to store information such as demoraphic information in medical and finance datasets \cite{yoon2020VIMEExtendingSuccess}
%heterogeneous tabular data are the most commobly used form of data and is essential for numerous applications \cite{borisov2022DeepNeuralNetworks}
%- tabular data is structured and usually presented a table with data points as rows and features as columns \cite{borisov2022DeepNeuralNetworks} \cite{yoon2020VIMEExtendingSuccess}


% Types of tabular data (Numerical, Categorical, etc.)
%tabular data is heterogeneous and contains a variety of attribute types (cont. and cat.) \cite{borisov2022DeepNeuralNetworks}
%and is very different to other data modalities like images, audio where only 1 feature is present \cite{borisov2022DeepNeuralNetworks}
%- Continuous \cite{lederrey2022DATGANIntegratingExperta}
%- Categorical 
%    - Binary
%    - Nominal (no order)
%    - Ordinal (order exists)
%\cite{lederrey2022DATGANIntegratingExperta}
%--> categorical are qualitative values not implying any numerical ordering and can take one out of limited set of values \cite{lane2003IntroductionStatistics}
%--> continuous are quantitative and "meassured in terms of numbers"
%- Dates/Timestamps \cite{hernandez2022SyntheticDataGeneration}
%- tabular data can be static (independent rows in a table) or dynamic (time series/multivariate time series) \cite{padhi2021TabularTransformersModeling}



% Challenges
% '- tabular data is not homogeneous which makes it challenging to work with for NNs \cite{borisov2022DeepNeuralNetworks}
% - correlations among features are weaker compared to homogeneous data (they have a spatial or semantic relationship) \cite{borisov2022DeepNeuralNetworks} \cite{yoon2020VIMEExtendingSuccess} % semantic relationship --> use embeddings to introduce relationship 

% data related pitfalls (noise, impreciseness, different attribute types and value ranges, missing values, privacy issues) \cite{borisov2022DeepNeuralNetworks}
% Mix of categorical and continouse data types \cite{li2021ImprovingGANInverse}

%Continouse columns usually follow complex distributions \cite{li2021ImprovingGANInverse}
%values in a dataset can be dependent on other variables \cite{lederrey2022DATGANIntegratingExperta} (also across multiple entries --> find example)
%high cardinality can lead to very sparse high-dimensional feature vectors and nonrobust models 

% biggest challenges when working with tabular data are: \cite{borisov2022DeepNeuralNetworks}
%     1. low-quality training data
%         - missing values
%         - outliers
%         - data errors/inconsistency
%         - expensive to collect
%         - class imbalanced \cite{li2021ImprovingGANInverse}
%     2. missing or complex irregular spatial dependencies
%         - no spatial correlation between variables
%         - dependencies between features are rather complex and irregular
%         - inductive bias not present --> cnns unsuited? %LOOKUP MEANING IN SOURCE
%     3. dependency on preprocessing
%         - performance depends strongly on preprocessing strategy \cite{gorishniy2022EmbeddingsNumericalFeatures}
%         - categorical preprocess especially challenge --> sparse feature matrix (one-hot) or synthetic ordering of unordered values (label enc) \cite{borisov2022DeepNeuralNetworks}
%         - information loss during (//due to the conversions?) \cite{fitkov-norris2012EvaluatingImpactCategorical}
%     4. Importance of single feature
%         - images class change only if several pixel change, tabular data 1 cell entity is sufficient for prediciton flip



\subsubsection{Tabular Data Preprocessing}
\label{sec:preprocessing}

Different data types can and should be processed into a meaningful format to be useful for deep learning models in different ways \cite{fan2020RelationalDataSynthesisa, lederrey2022DATGANIntegratingExperta}.
Data preprocessing itself consists of multiple different tasks: Data cleaning, data normalization, data transformation, data integration, missing value imputation and noise identification \cite{tritscher2020EvaluationPosthocXAIa}
While each of the preprocessing tasks is in itself important, 
\cite{fitkov-norris2012EvaluatingImpactCategorical} and \cite{gorishniy2022EmbeddingsNumericalFeatures} showed that a proper transformation of categorical and numerical entries respectively can have a significant influence on a deep learning models performance.
\cite{xu2019ModelingTabularData} showed the importance of normalization for synthetic tabular data generation.
Since the focus of the work is on tabular data and its synthesis, the following section will highlight the most important tabular data transformation and normalization approaches.

\subsubsubsection{Data Transformation}
\label{sec:dataTransformation}

\cite{borisov2022DeepNeuralNetworks} introduces a taxonomy for data transformation methods and subdivides the existing approaches into "Single-Dimensional Encodings" and "Multi-Dimensional Encodings".
The goal is to transform the different values column can take and transform them into a different (numeric) representation, such that it can be processed by a deep learning model.

\textbf{Single-Dimensional Encodings:}
Single-Dimensional encoding techniques encode each cell independently \cite{borisov2022DeepNeuralNetworks}.
The following approaches are common techniques to encode a categorical column entry, usually in a text format, into a numerical format.
In ordinal- (or label-) encoding a simple mapping from each category to a numeric value occurs. 
While this introduces a synthetic ordering of potentially unordered categories, one-hot encoding overcomes this issue by introducing a new vector with the length of all possible values a categorical column can take.
All values in this vector are assign to zero except one entry that represents the category that should be encoded, which is set to one.
However, this approach can lead to high dimensionality feature vectors if the cardinality of the unique categories in categorical columns is large.
Binary encoding tries to reduce the dimensionality by setting the vector length to a maximum of $log(c)$ for $c$ unique categorical values in a column.
Each possible value is mapped to a number like in ordinal-encoding starting at 0 but the number is represented as a binary vector.
The leave-one-out encoding technique is an approach to encode a categorical column based on the target column in a machine learning scenario. 
A categorical entry is replaced by the mean of the target variable of all rows where the same category is present, excluding the target value of the to be encoded value.
Lastly, a hash-based approach is worth mentioning, where a deterministic hash function transforms each category into a numerical form \cite{borisov2022DeepNeuralNetworks}.

Numerical data, such as integers or floating point numbers, can often be used directly in deep learning models without undergoing a special encoding process. 
This is because deep learning algorithms are designed to handle numerical data and can learn patterns and relationships within the data without the need for additional encoding.
However, \cite{gorishniy2022EmbeddingsNumericalFeatures} has shown that in some cases, encoding numerical data can improve the performance of deep learning models. 
Encoding numerical data can be achieved through various methods such as normalization (see \autoref{sec:dataNormalization}), discretization, or using embeddings.

Discretization techniques transform numerical features to categorical features. 
\cite{dougherty1995SupervisedUnsupervisedDiscretization} gives an overview on classical discretization techniques, such has equal interval width binning, where the continues values are divided and assign to certain amount of bins.
A modern approach by researchers from NVIDIA \cite{dong2022GeneratingSyntheticData} invented a tokenizer specifically for tabular data with float numbers. 
This tokenizer converts float numbers into a sequence of token IDs \cite{dong2022GeneratingSyntheticData}.

In Embedding techniques values that should be encoded (\eg words) are mapped to a vector representation. 
This vector of real numbers tries to capture "semantic regularities in vector spaces" \cite[p. 2]{pilaluisa2022ContextualWordEmbeddings}.
The goal of embeddings is to create a vector space, in which semantically similar values are also numerically similar \cite{pilaluisa2022ContextualWordEmbeddings}.
It can be differentiated between static embeddings and contextualized embeddings. 
While the former embedding technique always provides the same numerical representation for an input value, the latter embedding technique changes the vector representation of value based on its surrounding context \cite{pilaluisa2022ContextualWordEmbeddings}.
This is especially important in the natural language processing domain, where contextual embeddings has led to state-of-the-art improvements \cite{pilaluisa2022ContextualWordEmbeddings}.
Homonyms or polysemic words like "bat" or "second" are words that carry multiple different meanings and their semantic meaning therefore changes with the context they are used in. [TODO: QUELLE]
Hence, the contextualized embedding vector of the word "second" in the context of time (\eg "it took me 3 seconds") should be different to the one where "second" is used in the context of a competition (\eg "he achieved the second place").
Contextualized embeddings have to be learned during some form of (pre-) training \cite{devlin2019BERTPretrainingDeep, iida2021TABBIEPretrainedRepresentations, deng2021TURLTableUnderstanding}. 
Static embeddings, such as Word2Vec \cite{mikolov2013DistributedRepresentationsWords} are learned as well. 
It is also possible to use the embedding layers to get a vector representation without learning, which can be seen as a "feature tokenization" \cite{zheng2022DiffusionModelsMissing, gorishniy2021RevisitingDeepLearning}.
However, semantic similarity in a vector space cannot be achieved without any learning, so this embedding technique is more similar to a discretization/tokenization technique.


%HIER WEITERMACHEN
\cite{borisov2022DeepNeuralNetworks} provides a taxonomy for data learning for tabular data with "data transformation methods" into "single dim encodings" and "multi dim encodings" 
two dimensions:\cite{borisov2022DeepNeuralNetworks}
    1. single dimensional encoding
        - categorical encoding (ordinal, label, one-hot, binary encoding, leave-one-out-encoding, hash bashed) %Explanation for each in the paper
    2. multi-dimensional encoding 
        - vime trains encoder learns a representations of cat and num features into homogeneous representation
        - Supertml transforms tabular data into image format

\textbf{Multi-Dimensional Encodings:}
Multi-Dimensional encoding techniques focuses on encoding an entire record (or table-row) into another representation \cite{borisov2022DeepNeuralNetworks}.


\subsubsubsection{Data Normalization}
\label{sec:dataNormalization} 

numerical:
- min-max normalization
- GMM-based normalization


%- proper encoding of inputs (categorical in this paper) ahs a significant influence on models performance \cite{norris2012EvaluatingImpactCategorical}
%different Datatypes require different preprocessing \cite{fan2020RelationalDataSynthesisa} meaningfully \cite{lederrey2022DATGANIntegratingExperta}
%\cite{gorishniy2022EmbeddingsNumericalFeatures} shows that special encodings of numerical data influences the results of classification %vllt motivation? 
"the embedding step has a substantial impact on the model effectiveness, and its proper design can be a game-changer in tabular DL"
categorical:
- ordinal encoding -> assign int to each category
- one-hot -> binary vector for each category











\subsection{Synthetic Tabular Data Generation}

--> eacg data point as "row in table", or "sample from unknown joint distribution"
order of rows and columns does not carry any meaning \cite{somepalli2021SAINTImprovedNeural}


- Process or data driven methods \cite{goncalves2020GenerationEvaluationSynthetic}
    - Process driven
        - Simulation models \cite{kowalczyk2022TaxonomyUseSynthetic}
    - Data driven
        - Data Augmentation \cite{kowalczyk2022TaxonomyUseSynthetic}
        - statistical distributions \cite{kowalczyk2022TaxonomyUseSynthetic}
        - ML/DL- based Approaches \cite{kowalczyk2022TaxonomyUseSynthetic}

%challanges
interdependencies between variables must be captures \cite{lederrey2022DATGANIntegratingExperta}
danger of overfitting and generelize relationships between columns which only are present in training data but not in unseen data \cite{lederrey2022DATGANIntegratingExperta}
syn data generation can help imbalance problem \cite{borisov2022DeepNeuralNetworks}



\cite{zhao2022CTABGANEnhancingTabular} identifies four complex distributions that can occur when working with real world tabular data.
underlying properties of tabular data: \cite{zhao2022CTABGANEnhancingTabular}
- Single Gaussian variables:
    Single mode Gaussian distributions are very common.
    The distribution of real data is close to a single mode Gaussian distribution

Mixed data type variables:\cite{zhao2022CTABGANEnhancingTabular}
    a variable can be a mix of these two types + missing values. The Mortgage variable from the Loan dataset is a good example of mixed variable
    According to the data description, a loan holder can either have no mortgage (0 value) or a mortgage (any positive value). 
    In appearance, this variable is not a categorical type due to the numeric nature of the data. 
    So all 4 SOTA algorithms treat this variable as continuous type without capturing the special meaning of the value zero. 
    Hence, all 4 algorithms generate a value around 0 instead of exact 0. And the negative values for Mortgage have no/wrong meaning in the real world

Long tail distributions:\cite{zhao2022CTABGANEnhancingTabular}
    real world data can have long tail distributions where most of the occurrences happen near the initial value of the distribution, and rare cases towards the end
    Real data clearly has 99\% of occurrences happening at the start of the range, 
    but the distribution extends until around 25000. 
    In comparison none of the synthetic data generators is able to learn and imitate this behavior.

Skewed multi-mode continuous variables:\cite{zhao2022CTABGANEnhancingTabular}
    The term multimode is extended from Variational Gaussian Mixtures (VGM)
    This is not a typical Gaussian distribution. There is an obvious peak
    with several other lower peaks
    This behavior is difficult to capture for the SOTA data generator


%-------------------------------------------------------------------------
\section{Deep Learning Architectures}
\label{ch:preliminaries-deepLearningArchitectures}

\subsection{Neural Networks}
\label{ch:preliminaries-deepLearningArchitectures-neuralNetworks}

- success of NN \& use of special architectures (rnn, cnn, etc.) in variety of domains \cite{borisov2022DeepNeuralNetworks}
- DL works well for homogeneous data \cite{borisov2022DeepNeuralNetworks}
% brief overview of neural networks (perceptron, multilayer perceptron)
% special neural networks 
% --> convolutional neural networks
% --> recurrent neural networks
% --> Residual neural networks
% --> Attention mechanism

\section{Generative Algorithms}
\label{ch:preliminaries-generativeAlgorithms}

\subsection{Autoencoders}
\label{ch:preliminaries-generativeAlgorithms-variationalAutoencoders}

from \cite{kingma2013AutoEncodingVariationalBayes}

Figure 1 from \cite{razghandi2022VariationalAutoencoderGenerativea}

% what are autoencoders
encoder transforms input into latent space which is reconstructed by decoder \cite{razghandi2022VariationalAutoencoderGenerativea}
--> suffer from "lack of regularity in latent space \cite{razghandi2022VariationalAutoencoderGenerativea}

% what are variational autoencoders
use the KL divergence and encode a gaussian distribution in latent space \cite{razghandi2022VariationalAutoencoderGenerativea}
% how do they work
% mathematical formulation
% Advantages / Problems / Challenges


\subsection{Generative Adversarial Networks}
\label{ch:preliminaries-generativeAlgorithms-generativeAdversarialNetworks}

% what are GAN's
\cite{goodfellow2020GenerativeAdversarialNetworks}
- have been used very sucessfully in many different domains \cite{li2022TTSGANTransformerbasedTimeSeries}
 

% how do they work
\cite{zhao2022CTABGANEnhancingTabular}:
- trained via a zero-sum min-max game 
- the discriminator tries to maximize the objective, while the generator tries to minimize it.
- mentor (D) providing feedback to a student (G) on the quality of his work


\cite{li2022TTSGANTransformerbasedTimeSeries}
- 2 NN (gen and dis)
- gen inp: rand vec of specified dimension; gen out: same dimension, as similar to real training data
- dis: binary classifier, distiguish real and generated data
--> play zero sum game against each other
--> trz to each nash equilibrium

% what improvements have been done to GAN's
- gans struggled with generating discrete variables \cite{torfi2020CorGANCorrelationCapturingConvolutionala}
- State of the art Gan approaches only focused on continouse and categorical types, overlooking mixed data types \cite{zhao2022CTABGANEnhancingTabular}

- conditional GAN \cite{mirza2014ConditionalGenerativeAdversarial}


% mathematical formulation
% Advantages / Problems / Challenges (Mode Collapse, etc.)
- remarkable performance generating syntehtic images and time series data \cite{mckeever2020SynthesisingTabularDatasets}
- struggle with mode collapse --> generate same sample \cite{torfi2020CorGANCorrelationCapturingConvolutionala}
- wasserstein loss helps against mode collapse \cite{frogner2015LearningWassersteinLoss} \cite{arjovsky2017WassersteinGenerativeAdversarial} and has been implemented in many gans (e.g. \cite{zhao2022CTABGANEnhancingTabular})
- gradient penalty \cite{gulrajani2017ImprovedTrainingWasserstein}


\subsection{Transformers}
\label{ch:preliminaries-generativeAlgorithms-transformers}
relies on multiple self-attention layers, surpasses other network architectures and shows properties of universal computation engine \cite{li2022TTSGANTransformerbasedTimeSeries}

have been very successfull on textual and visual data (//siehe quellen im paper) \cite{borisov2022DeepNeuralNetworks} and also been applied to tabular data \cite{padhi2021TabularTransformersModeling} \cite{gorishniy2022EmbeddingsNumericalFeatures}



% What are transformers
% How do they work
% In what context are they used (usually not for data synthesis)
% Advantages / Problems / Challenges

- TabNet is one of the first transformer-based models for tabular data \cite{borisov2022DeepNeuralNetworks}
- padhi2021TabularTransformersModeling

\subsection{Diffusion Probabilistic Models}
\label{ch:preliminaries-generativeAlgorithms-diffusionProbabilisticModels}

% What are diffusion probabilistic models
first paper \cite{sohl-dickstein2015DeepUnsupervisedLearning}
first famouse paper \cite{ho2020DenoisingDiffusionProbabilistic}
improvements> \cite{nichol2021ImprovedDenoisingDiffusion}
follow up: \cite{dhariwal2021DiffusionModelsBeat}

\cite{ho2022ClassifierFreeDiffusionGuidance}

\cite{rombach2022HighResolutionImageSynthesis}

% mathematical formulation
% How do they work
% In what context are they used (usually image synthesis)
% Advantages / Problems / Challenges


% for tabular data
\cite{zheng2022DiffusionModelsMissing} for missing data

\cite{kotelnikov2022TabDDPMModellingTabular} tabddpm
\cite{hoogeboom2021ArgmaxFlowsMultinomial} multinomial diffusion


\subsection{Diffusion Probabilistic Models for Tabular Data}
\label{ch:preliminaries-generativeAlgorithms-diffusionProbabilisticModelsTabularData}

% How can Diffusion Probabilistic Models be used for tabular data
% Challenges of using Diffusion Probabilistic Models for tabular data (mixed data types --> different noising process, etc.)


%-------------------------------------------------------------------------
\section{Evaluation of Synthetic Tabular Data}
\label{ch:preliminaries-evaluationOfSyntheticTabularData}

- there is no universal metric for data synthesis \cite{hernandez2022SyntheticDataGeneration}
- utility and information disclosure metric dimensions \cite{goncalves2020GenerationEvaluationSynthetic}
- structural similarity \cite{elemam2020SevenWaysEvaluate}

\subsection{Statistical Evaluation}
\label{ch:preliminaries-evaluationOfSyntheticTabularData-statisticalEvaluation}

\subsection{Machine Learning Efficiency}
\label{ch:preliminaries-evaluationOfSyntheticTabularData-machineLearningEfficiency}

\subsection{Privacy Evaluation}
\label{ch:preliminaries-evaluationOfSyntheticTabularData-privacyEvaluation}

\subsection{Additional Evaluation Methods}
\label{ch:preliminaries-evaluationOfSyntheticTabularData-otherMetrics}

% Bias and stability
% Domain Expertise

\subsection{Similarity Score}
\label{ch:preliminaries-evaluationOfSyntheticTabularData-similarityScore}
% https://www.researchgate.net/publication/344227988_On_the_Generation_and_Evaluation_of_Tabular_Data_using_GANs

% TOREAD: https://www.researchgate.net/publication/361949372_TabSynDex_A_Universal_Metric_for_Robust_Evaluation_of_Synthetic_Tabular_Data

%-------------------------------------------------------------------------





\chapter{Related Work}
\label{ch:relatedWork}


%-------------------------------------------------------------------------
\section{Generative Adversarial Networks Models}
\label{ch:relatedWork-generativeAdversarialNetworksModels}


mirza2014ConditionalGenerativeAdversarial
% what improvements have been done to GAN's
- gans struggled with generating discrete variables \cite{torfi2020CorGANCorrelationCapturingConvolutionala}
- State of the art Gan approaches only focused on continouse and categorical types, overlooking mixed data types \cite{zhao2022CTABGANEnhancingTabular}

- conditional GAN \cite{mirza2014ConditionalGenerativeAdversarial}


% mathematical formulation
% Advantages / Problems / Challenges (Mode Collapse, etc.)
- remarkable performance generating syntehtic images and time series data \cite{mckeever2020SynthesisingTabularDatasets}
- struggle with mode collapse --> generate same sample \cite{torfi2020CorGANCorrelationCapturingConvolutionala}
- wasserstein loss helps against mode collapse \cite{frogner2015LearningWassersteinLoss} \cite{arjovsky2017WassersteinGenerativeAdversarial} and has been implemented in many gans (e.g. \cite{zhao2022CTABGANEnhancingTabular})
- gradient penalty \cite{gulrajani2017ImprovedTrainingWasserstein}


\subsection{CTGAN}
\label{ch:relatedWork-generativeAdversarialNetworksModels-ctgan}

\subsection{CTAB GAN}
\label{ch:relatedWork-generativeAdversarialNetworksModels-ctabGAN}
%-------------------------------------------------------------------------
\section{Transformer Models}
\label{ch:relatedWork-transformers}

TabTransformer only embeddeds categorical values --> cont not through attention --> no correlation capture \cite{somepalli2021SAINTImprovedNeural}

- TabNet is one of the first transformer-based models for tabular data \cite{borisov2022DeepNeuralNetworks}
- padhi2021TabularTransformersModeling


%-------------------------------------------------------------------------
\section{Diffusion Models}
\label{ch:relatedWork-diffusionModels}


\subsection{Diffusion Probabilistic Models for Tabular Data}
\label{ch:preliminaries-generativeAlgorithms-diffusionProbabilisticModelsTabularData}

% How can Diffusion Probabilistic Models be used for tabular data
% Challenges of using Diffusion Probabilistic Models for tabular data (mixed data types --> different noising process, etc.)

\subsection{Tab-DDPM}
\label{ch:relatedWork-diffusionModels-tabDDPM}

% Paper for image synthesis
% paper for diffusion for missing data entries
% Paper for tabular data synthesis


%-------------------------------------------------------------------------

\chapter{Conceptual Design}
\label{ch:conceptualDesign}

\section{Requirements}
\label{ch:requirements}

In this chapter, the requirements for the synthetic data generation system will be first established.
The goal is to gather a systematic collection of requirements on which the success of the developed system can be measured.
Firstly, non-functional requirements will be explained before continuing with the functional requirements.


\subsection*{Non-Functional Requirements}

Non-functional requirements (or qualitative requirements) describe requirements and constraints on a system that governs how the functions specified in the functional requirements are to be performed \cite{broy2021EinfuehrungSoftwaretechnik}.
They focus on the question of how a system's functional requirements should be realized \cite{broy2021EinfuehrungSoftwaretechnik}.
The ISO-Norm 25010 \cite{iso/iecSystemsSoftwareEngineering} defines the base characteristics relevant for a software system's quality \cite{haoues2017GuidelineSoftwareArchitecture}.
The key elements of software architectures include functional suitability, reliability, performance efficiency, compatibility, usability, security, maintainability, and portability \cite{haoues2017GuidelineSoftwareArchitecture}.
Each of these is of different importance for each software system.
Additionally, \cite{vogelsang2019RequirementsEngineeringMachine} highlights that additional quality requirements might be necessary in the context of machine learning tasks, for which a standard has yet to be defined.
These could include (but are not limited to) quantitative targets (\ie measurable results), explainability, freedom from discrimination, legal and regulatory requirements, and data requirements.

This thesis focuses on a research prototype rather than a fully developed software system.
Therefore, only a subset of the aforementioned requirements are prioritized during development.
The following requirements R1 - R5 are the non-functional requirements for the software developed in this thesis:

\begin{description}
	\item[R1 - Functional Suitability:]
		Functional Suitability refers to "the degree to which a product or system provides functions that meet stated and implied needs when used under specified conditions" \cite[p. 219]{bass2013SoftwareArchitecturePractice}.
		In the context of this thesis, the system first needs to be able to generate synthetic data and needs to be able to show how different tabular processing techniques influence the performance of the overall synthesis approach.
		This performance should be measured across different similarity measures.
	\newpage
	\item[R2 - Maintainability:]
		Maintainability focuses on the "degree of effectiveness and efficiency with which a product or system can be modified by the intended maintainers" \cite[p. 220]{bass2013SoftwareArchitecturePractice}.
		One of the core questions of this thesis is to compare different tabular processing techniques in diffusion \glspl{model}.
		For this, a variety of different approaches are intended to be compared.
		Hence, it is required that the code and software architecture is designed in a maintainable way, such that it can be easily extended with additional processing techniques, not only by the developer,
		but also for future researchers who might build upon the work of this thesis.

	\item[R3 - Performance efficiency:]
		In the context of deep learning, the program's performance is always essential since training, hyperparameter search, and inference usually demand a lot of computation time by nature.
		Consequently, the developed software should reduce unnecessary computations as much as possible.

	\item[R4 - Portability/Reproducibility:]
		Portability is usually referring to the extent of "effectiveness and efficiency with which a system, product, or component can be transferred from one hardware, software,
		or other operational or usage environment to another" \cite[p. 220]{bass2013SoftwareArchitecturePractice}.
		Since this thesis should encourage other researchers to reproduce and extend the developed codebase, the software should be able to run on other machines running a different operating system given a specified development environment,
		including packages used and their version numbers.
		In general, it is required that all results should be fully reproducible, given the parameters and the configuration of the experimental setups

	\item[R5 - Quantitative Targets:]
		It is required that the different \gls{model} versions produced in the thesis are compared using metrics that are commonly used in the domain of tabular data synthesis.
		This should enhance the comparability to other approaches.
\end{description}


\subsection*{Functional Requirements}
\label{sec:func_requirements}
Functional requirements describe what the software system should do and which practical aspects must be implemented.
Please note that the software of \cite{kotelnikov2022TabDDPMModellingTabular} will be extended, which has consequences for certain functional requirements.

\begin{description}
	\item[FR1 - Data Import and Export:]
		The software shall support importing tabular data in the same format as specified in \cite{kotelnikov2022TabDDPMModellingTabular}.
		\textcite{kotelnikov2022TabDDPMModellingTabular} makes use of datasets, that are already separated into numerical, categorical, and target columns, each saved in a NumPy \cite{harris2020array} array.
		Additionally, the datasets that will be imported have to be separated into training, validation, and testing sets.
		For the NumPy arrays to be imported, they need to be of file type ".npy".
		Synthetic data produced by \glspl{model} shall be saved in the same data segmentation format as the imported data.

	\item[FR2 - Tabular Processing:]
		The software shall support tabular data processing techniques that are able to encode and decode tabular data.
		An \textit{IdentityProcessor} that has no effect is required to be implemented to test \glspl{model} without any tabular processing mechanism.
		Additional data transformation techniques proposed by \cite{kotelnikov2022TabDDPMModellingTabular}, including normalization or categorical encoding, shall still be supported as intended by the authors.
		Adding new tabular processing mechanisms shall not affect other existing tabular processing mechanisms. This requirement implies several sub-requirements: 

		\item[FR2.1 - Tabular Processing Fit:]
		If necessary, a Tabular processing mechanism should be able to be fitted according to the data.
		However, it is required that the tabular processing mechanism only receives training and validation data and does not have access to the test data.

		\item[FR2.2 - Tabular Processing Transform:]
		Each Tabular processing mechanism needs to transform the raw data into a specified format.

		\item[FR2.3 - Tabular Processing Inverse Transform:]
		Each Tabular processing mechanism \\needs to be able to inversely transform data back into its original form.

		\item[FR3 - Diffusion Process Configuration:]
		Several experiments with different \gls{model} configurations will be tested.
		The software shall allow users to configure the diffusion process by specifying parameters such as the number of iterations, batch size, sampling size, and others.

		\item[FR4 - Training, Tuning, Sampling, Evaluation:]
		The software shall support the training of various \gls{model} versions, including the TabDDPM diffusion \gls{model}, variations of TabDDPM and other non-diffusion baseline \glspl{model}.
		The software shall allow finding of suitable hyperparameters in a specified hyperparameter search space.
		During training, relevant metrics and loss values shall be saved.
		The software shall generate synthetic samples from the trained diffusion \gls{model} during the sampling process, where the user specifies the number of synthetic samples to be generated.
		The software shall handle the evaluation of synthetic data in multiple ways.
		Firstly, the existing evaluation framework of \cite{kotelnikov2022TabDDPMModellingTabular} shall remain.
		Additionally, a similarity evaluation shall be implemented according to \cite{chundawat2022UniversalMetricRobust}.
		Lastly, the software should also produce visualizations of certain properties of the synthetic data to allow a visual comparison with the same properties of the real data, for example, to allow for a comparison of distributions of columns.

	\item[FR5 - Modularity:]
		The individual software components (tabular data processing, training, sampling, evaluation) shall be implemented in such a way,
		that they are easily replaceable by alternatives (for tabular data processing) or shall be executed individually (training, sampling, evaluation).
		For example, sampling synthetic data shall be separated from the training so that not every sampling process requires training a \gls{model} from the start.
\end{description}


\section{Existing Code Base}
\label{ch:conceptualDesign-existingCodeBase}

The software \cite{akim2023TabDDPMModellingTabular} used in this experiment was programmed in the programming language python \cite{van1995python} and is based upon the work of \textcite{kotelnikov2022TabDDPMModellingTabular}.
It is changed and modified according to the needs of the experiments.
This section will first explain the architecture of the existing code base before introducing the proposed adaptations in the next section.

\begin{figure}[h]
	\centering
	\includegraphics[width=0.8\textwidth]{images/Overall_original.png}
	\caption[Overview Original Software]{Overview synthetic data generation process in \cite{akim2023TabDDPMModellingTabular}}
	\label{fig:overall_original}
\end{figure}


\subsection{Original Implementation}
\label{ch:conceptualDesign-existingCodeBase-originalImplementation}

\textcite{akim2023TabDDPMModellingTabular} provides a software implementation for their proposed TabDDPM.
This implementation consists of multiple scripts and implements several of the functional requirements listed in \Autoref{sec:func_requirements}, including FR1, FR4, FR5, and parts of FR4.
This section will explain what kind of software was already provided by \cite{kotelnikov2022TabDDPMModellingTabular} and explain important architectural design choices.

\subsubsection[]{Software Procedure}
\label{ch:Software_Procedure}

An overview of how the overall process in this synthetic data generation approach is done can be seen in \Autoref{fig:overall_original}.
The whole software process can be roughly summarized into three parts, the training, the sampling, and the evaluation.
In the training, real training data is taken from the original dataset.
This training data is preprocessed, according to a predefined preprocessing strategy that may vary from dataset to dataset (compare \Autoref{sec:preprocessing}).
After preprocessing, the diffusion \gls{model} is initialized and trained to learn the underlying joint distribution of the training data.
Once the training is finished, the sampling of new synthetic data can be executed.
The previously trained \gls{model} is loaded and synthetic data will be produced by denoising previously sampled noise.
The synthetic data will be in the same format as the training data after it has been preprocessed.
Hence, the synthetic data will be postprocessed by inverting the preprocessing, such that the synthetic data will be in the same format as the raw training data.
As a last step, the synthetic data will be evaluated. 
For this, the held-back real test set will be loaded.
With the real test set and the synthetic dataset sharing the same data format, a machine learning efficacy evaluation (compare \Autoref{ch:preliminaries-machineLearningEfficacy}) is performed to compare the two datasets.
Ultimately, the results of this evaluation are saved and can be interpreted by the user.

\subsubsection[]{Scripts}
\label{ch:scripts}

The implementation \cite{akim2023TabDDPMModellingTabular} already provides the necessary scripts that allow for an
easy training, sampling, evaluation, and hyperparameter tuning for their proposed TabDDPM \gls{model}.
Additional non-diffusion baseline \glspl{model} the TabDDPM is compared against are supported as well.
The functionality of the most important scripts can be summarized in the following way:

\begin{description}
	\item[train.py:]
		The script "train.py" is responsible for the training of the diffusion \gls{model}. 
		It accepts configuration parameters that guide the training process. 
		Initially, it loads the designated dataset and carries out preprocessing as per the given configuration. 
		Subsequently, it initializes the necessary class instances and starts the training loop. 
		Once the training loop is finished, the script saves the trained \gls{model} along with its loss history.

	\item[sample.py:]
		The script "sample.py" carries out the task of sampling from a pre-trained diffusion \gls{model}. 
		It takes in specific configuration parameters that are used during the sampling procedure. 
		This script starts by loading a pre-trained \gls{model} and generating samples from it. 
		These new samples are then modified in accordance with the inverse of the preprocessing defined in the configuration.
		Once the transformation is complete, the resulting samples are saved.

	\item[\text{eval\_[catboost\textbar mlp\textbar simple].py:}]
		The scripts "eval\_catboost.py", "eval\_mlp.py", and \\"eval\_simple.py" evaluate a synthetic dataset based on machine learning efficacy. 
		In this process, a machine learning \gls{model}, chosen by the user, is trained on either real or synthetic data, as specified in the configuration. 
		The \gls{model}'s performance is then assessed on the real test set by computing a set of metric scores.

	\item[pipeline\_*.py\footnotemark:]
		The scripts, denoted as "pipeline\_*.py", define the full processing pipeline, which includes the processes of training, sampling, and evaluation. 
		They ensure that each function within the pipeline correctly receives its necessary attributes from the configuration. 
		These scripts also provide the flexibility to execute specific portions of the pipeline individually.
		\footnotetext[1]{Each implemented baseline \gls{model} (SMOTE, TVAE, CTABGAN, CTABGAN+) has its own implementation as pipeline\_\textit{modelName}.py.}

	\item[tune\_*.py\footnotemark:]
		The "tune\_*.py" scripts are generally responsible for the hyperparameter tuning process.
		It starts by defining a hyperparameter search space (see \cite[Table 1, p. 4]{kotelnikov2022TabDDPMModellingTabular} and \cite[Table 7-11, p. 13 f.]{kotelnikov2022TabDDPMModellingTabular}).
		Next, an objective function is defined that is maximized for 50 trials through the Optuna framework \cite{optuna_2019}.
		Within the objective function, a set of hyperparameters is selected from the search space.
		These selected hyperparameters are used to invoke \textit{pipeline.py}, which initiates the \gls{model} training
		Afterward, for five different random initializations, \textit{pipeline.py} is called to sample and evaluate the \gls{model}, creating, and assessing five different synthetic dataset versions.
		In each training, sampling, and evaluation, the training and validation set are redistributed and shuffled to implement some form of cross-validation \cite{kohavi1995StudyCrossvalidationBootstrap}.
		The average evaluation score (machine learning efficacy based) of the five synthetic dataset versions is returned as the objective that Optuna tries to maximize.
		After optimization, the best hyperparameter configuration is saved.
		If the user has set an \textit{- - eval\_seeds} flag, a final evaluation of the best-found \gls{model} will be started by calling \textit{eval\_seeds.py}.
		\footnotetext[2]{Each implemented baseline \gls{model} has its own implementation as tune\_\textit{modelName}.py.}
		
	\item[eval\_seeds.py:]
		The "eval\_seeds.py" script's purpose is to perform an extensive evaluation, given a trained \gls{model}.
		The script starts by loading the pre-trained sampling \gls{model} ([TabDDPM|SMOTE|CTABGAN|CTABGAN+|TVAE]).
		Given the sampling \gls{model}, $n\_datasets$ synthetic datasets will be produced depending on the specified evaluation parameters by calling the sampling script in the respective \textit{pipeline\_*.py} script.
		For each produced dataset, evaluation is performed using a predefined evaluation \gls{model} ([Catboost|MLP]) for a predefined number of random initializations.
		Ultimately, the average metrics are calculated over all performed runs, reported, and saved.
\end{description}

In summary, to get a fully optimized diffusion \gls{model} with an extensive evaluation, the user just has to run the \textit{tune\_ddpm.py} script with the \textit{- - eval\_seeds} flag.
A detailed activity diagram for each script can be found in the Appendix \ref{A:activity_diagrams}.

\subsubsection[]{Configuration}

In order to start one of the above scripts, a configuration file has to be specified, detailing the most important parameters for training, sampling, and evaluation.
For example, \Autoref{lst:configuration} shows how such a configuration file could look like and which parameters need to be specified.
The file is stored as a \gls{toml} \glsadd{tomlaccr} file (short for "Tom's Obvious, Minimal Language") \cite{preston-werner2021TOMLTomObviousa}, which contains sections, indicated by squared brackets. 
Within these sections, key-value pairs define configuration options.

\begin{lstlisting}[label={lst:configuration}, caption={Example Configuration File}]
    parent_dir = "exp/adult/check"
    real_data_path = "data/adult/"
    num_numerical_features = 6
    model_type = "mlp"
    seed = 0
    device = "cuda:0"

    [model_params]
    num_classes = 2
    is_y_cond = true

    [model_params.rtdl_params]
    d_layers = [
        256,
        256,
    ]
    dropout = 0.0

    [diffusion_params]
    num_timesteps = 1000
    gaussian_loss_type = "mse"
    scheduler = "cosine"

    [train.main]
    steps = 1000
    lr = 0.001
    weight_decay = 1e-05
    batch_size = 4096

    [train.T]
    seed = 0
    normalization = "quantile"
    num_nan_policy = "__none__"
    cat_nan_policy = "__none__"
    cat_min_frequency = "__none__"
    cat_encoding = "__none__"
    y_policy = "default"

    [sample]
    num_samples = 216000
    batch_size = 10000
    seed = 0

    [eval.type]
    eval_model = "catboost"
    eval_type = "synthetic"

    [eval.T]
    seed = 0
    normalization = "__none__"
    num_nan_policy = "__none__"
    cat_nan_policy = "__none__"
    cat_min_frequency = "__none__"
    cat_encoding = "__none__"
    y_policy = "default"
\end{lstlisting}


The key \textit{model\_type} (line 4) refers to the \gls{model} that should approximate the noise in the reverse noising process.
Most parameters are self-explanatory, controlling one specific part of the pipeline.
For example, all \textit{model\_params} (lines (ll.) 8-10) specify different relevant aspects for instantiating the \gls{model}.
Inside the \textit{model\_params} section, the \textit{num\_classes} key indicates the number of possible target classes the given dataset has, and \textit{is\_y\_cond} indicates if the dataset is a classification (=\textit{True}) or regression dataset (=\textit{False}).
The \textit{model\_params.rtdl\_params}\footnote{rtdl is an abbreviation for the python library \cite{gorishniy2021RevisitingDeepLearning} from which the \gls{mlp} \gls{model} architecture was used.} (ll. 12-17) specify the \glp{mlp} architecture, by controlling the number of layers and the amount of dropout.
As the name suggests, the keys inside the \textit{diffusion\_params} (ll. 19-22) section control the diffusion process.
Noteworthy are the \textit{train.T} (ll. 30-37) and \textit{eval.T} (ll. 48-55) sections, whose keys define how preprocessing is done at the beginning of training and evaluation, respectively.
The \textit{eval.type} section keys are useful to further specify the kind of evaluation that will be executed.
The machine learning efficacy \gls{model} is determined by the \textit{eval\_model} key and is, in this example, set to the CatBoost \gls{model}.
To control what kind of data will be evaluated, the \textit{eval\_type} key can be set to either \textit{"real"} or \textit{"synthetic"}.
If the value is set to \textit{"real"}, the real training data is loaded and compared to the real test data, as if it was synthetic data.
This setting allows the creation of baseline metric scores against which synthetic values are compared.
As this is usually done only once per dataset, \textit{eval\_type} is usually set to \textit{"synthetic"} to calculate metric scores for the newly created synthetic data.
This configuration structure allows to efficiently perform a variety of experiments with different key configurations by simply changing the configuration file and inputting it into the next experimental run.

\subsubsection[]{Dataset Segmentation}
\label{sec:data_format}
The authors provide access to 12 different datasets, which have already been cleaned and formatted by a program provided by \textcite{gorishniy2023EmbeddingsNumericalFeatures}.
In order to use and process the data, the software of \cite{kotelnikov2022TabDDPMModellingTabular} is built towards processing the data that was segmented as specified by \textcite{gorishniy2023EmbeddingsNumericalFeatures}.
For the purpose of clarity in this thesis, any future reference to "data segmentation" will refer to the separation of a dataset into multiple parts, detailed as follows:

Each dataset contains data separated into multiple subcomponents, saved as individual files.
To create these subcomponents, the columns of the dataset are manually classified into categorical, numerical, (\textit{"X"}) and target (\textit{"y"}) columns.
Secondly, the dataset records are distributed into a training, validation, and testing set.
Therefore, each dataset is separated into nine parts, conforming to the subsequent naming convention:
\textit{X\_[cat|num]\_[train|val|test].npy} and for the target \textit{y\_[train|val|test].npy}

Additionally, a \textit{info.json} file is provided, containing the necessary information for handling the data.
In \Autoref{lst:info}, one can see how such an info-file contains the information on how the splits mentioned above have been done and what kind of task this dataset usually has (it is differentiated between binary classification (binclass), multiclass classification (multiclass), and regression (regression)).
\begin{lstlisting}[label={lst:info},caption={Example Data-info File}]
    {
    "name": "Adult",
    "id": "adult--default",
    "task_type": "binclass",
    "n_num_features": 6,
    "n_cat_features": 8,
    "test_size": 16281,
    "train_size": 26048,
    "val_size": 6513
    }
\end{lstlisting}
Inside the code of TabDDPM \cite{akim2023TabDDPMModellingTabular}, a custom dataset class provided by the author handles any dataset that follows the above naming convention and provides a matching information file.

\section{Proposed Concept}
\label{ch:conceptualDesign-changes}

Changes to the current architecture have been made to answer the research questions specified in \Autoref{ch:intro-goals}.
\Autoref{fig:Overall_changed} shows that the overall synthesis process does not deviate much from \Autoref{fig:overall_original}.
Changes are highlighted in green and affect each step in the diffusion process: the training, sampling, and evaluation step.
For a detailed overview of how each individual scripts changed, please refer to Appendix \ref{A:activity_diagrams}.

\begin{figure}[h]
	\centering
	\includegraphics[width=0.8\textwidth]{images/Overall_changed.png}
	\caption[Overview Software Changes]{Overview of proposed synthetic data generation process changes \\(highlighted in green)}
	\label{fig:Overall_changed}
\end{figure}

\begin{description}
	\item[C1-Tabular Processing:] To test the effect of different tabular processing techniques, a "Tabular processing" step is required before training the diffusion \gls{model}.
		Within this processing, the raw data is transformed according to the defined tabular processing strategy.
		The tabular processing data output format needs to be the same as before the transformation, described in \Autoref{sec:data_format}, to ensure
		that the remaining pipeline remains functional and keeps changes to the original code minimal.
	\item[C2-Inverse Tabular Processing:] Similar to the data preprocessing, which requires an inverse data postprocessing, to turn the data back into its original format and scale, the same is necessary for the tabular processing.
		Since the \gls{model} is trained on data that has been transformed according to the tabular processing mechanism, the diffusion \gls{model} will produce data in the same format.
		Consequently, the synthetic data must be transformed back by an inverse function of the tabular processing mechanism before it is evaluated.
	\item[C3-Similarity Evaluation:] Lastly, an extended evaluation of the synthetic data's similarity to the real data should be performed.
		For this, the original machine learning efficacy evaluation remained unchanged.
		However, the evaluation pipeline is extended by a similarity evaluation that does not only calculate a unified score, TabSynDex \cite{chundawat2022UniversalMetricRobust}, but also reports other metric results that are of interest.
		Hyperparameters of the generation \glspl{model} can also be tuned towards the TabSynDex score.
		In addition to the numeric metric results, several visualizations comparing real and synthetic data will be generated.
\end{description}

%<-------------
\subsection{Tabular Processing Criteria}
\label{ch:Concept-criteria}

The main source for finding the processing mechanisms should be academic literature, where said mechanism has been successfully applied to a related problem.
The selection of which tabular processing mechanisms are realized heavily depends on the requirements defined in \Autoref{sec:func_requirements}.

For a tabular processing mechanism to be used, the following conditions must be met:

\begin{description}
	\item[1. Reversibility:] The tabular processing mechanism must not only transform the data into a different format, but it also must be able to revert the transformation
	      such that the data can be transformed back into its original form.
	\item[2. Compatibility:] The data must be able to be separated into numerical and categorical parts after the transformation
	      since the remaining code expects data in this format, as specified in \Autoref{sec:data_format}.
	\item[3. Complexity:] The time it takes to transform the data should be within a reasonable timeframe. They shall not add too much computation time to the already extensive \gls{model}-tuning process.
	\item[4. Availability:] Due to the limited time frame of this thesis, the code for the mechanisms should be publicly available (or could be implemented within a reasonable timeframe).
\end{description}

\subsection[]{Evaluation}
\label{ch:conceptualDesign-Evaluation}
The evaluation framework shall contain multiple evaluation aspects.
Firstly, the existing machine learning efficacy using a CatBoost \gls{model} proposed by \cite{kotelnikov2022TabDDPMModellingTabular} shall be executed and remain untouched.
This is important to ensure that a replication of the original authors' results remains reproducible.

Secondly, the TabSynDex \cite{chundawat2022UniversalMetricRobust} similarity consisting of multiple similarity metrics will be performed in each evaluation.
The use of this similarity metric helps to provide a broader perspective on synthetic data quality, beyond just evaluating machine learning performance.

Machine learning efficacy that is based upon an \gls{mlp} or using a set of classification \glspl{model} will not be performed for the following reasons:
\begin{enumerate}
	\item \textcite{kotelnikov2022TabDDPMModellingTabular} argues convincingly that machine learning efficacy not based on CatBoost is less informative compared to the CatBoost counterpart.
	\item The TabSynDex metric contains a machine learning efficacy metric based upon a set of standard machine learning \glspl{model}.
\end{enumerate}
Consequently, machine learning efficacy will be evaluated twice in each evaluation, once based on CatBoost and once inside the TabSynDex metric using a set of standard machine learning \glspl{model}.

Lastly, for a visual evaluation, several plots indicating the similarity of the synthetic data to the real data shall be generated using the implementation of \textcite{brenninkmeijer2019GenerationEvaluationTabular}.
The plots shall effectively highlight crucial characteristics inherent in the real dataset, which need to be accurately mirrored in the synthetic dataset.
If necessary, modifications to the plots may be required.

\chapter{Methodology}
\label{ch:methodology}

This chapter delves into the methodology employed to address the research questions and objectives outlined in the previous chapters.
The chapter is organized into three main sections.
\Autoref{ch:architecture} discusses the architecture of our proposed solution, elaborating on the tabular processor and its implementations, namely the BGMProcessor and FTProcessor.
In addition, this section also highlights the modifications made to the existing software to incorporate the proposed solutions.

\Autoref{ch:methods-experimentalSetup} focuses on the experimental setup, outlining the experiments conducted to evaluate the performance and effectiveness of the proposed models.
It provides detailed information on the execution environment and the hyperparameter search space used in the study, ensuring the reproducibility of the results.

Finally, \Autoref{ch:methods-datasets} presents the dataset utilized in the experiments, including its source and characteristics.
This information is crucial to understanding the context in which our proposed methodology was tested and evaluated.

By the end of this chapter, the reader will gain a comprehensive understanding of the methodology implemented in this thesis, which will lead to the presentation and analysis of the results in the subsequent chapters.

\section{Architecture}
\label{ch:architecture}

The software's architecture was developed with the requirements in \Autoref{ch:requirements} in mind.
The following section will explain how the individual additions to the existing code are designed and why the design was chosen.


\subsection{Tabular Processor}
\label{ch:architecture-tabularProcessor}

To allow for flexible use and the possibility of adding additional processing mechanisms (R5 - Maintainability), the \textit{strategy design} pattern (\Autoref{fig:design}) was chosen \cite{gamma1994design}.
The pattern is beneficial if a specific behavior is required, but needs to be realized in different ways.
The \textit{Strategy} class defines what kind of methods need to be implemented and what behavior the class should have.
The \textit{Strategy} classes children define a \textit{ConcreteStrategy} and realize the desired behavior in different ways.
Different \textit{ConcreteStrategies} can be used interchangeably independent of the \textit{Context} they are used in since all implement the same functionalities.
The \textit{Context} is has a reference to the \textit{ConcreteStrategies} and ultimately decides when an instance of type \textit{Strategy} is executed.
As the \textit{ConcreteStrategies} implement the same interface, the \textit{Context} can execute any function a \textit{ConcreteStrategies} has implemented, without needing to know the inner workings of that function.
Hence, whenever the \textit{Context} needs to execute a certain functionality, it delegates the execution to the \textit{ConcreteStrategy} object it needs, which may vary on the situation.
One of the biggest advantages is, that this way new \textit{ConcreteStrategies} can be added easily without affecting any of the other \textit{ConcreteStrategy} or the \textit{Context} \cite{gamma1994design}.

\begin{figure}[h]
	\centering
	\includegraphics[width=0.8\textwidth]{images/strategy.png}
	\caption[Strategy Design Pattern]{Strategy design pattern \cite[p. 316]{gamma1994design}}
	\label{fig:design}
\end{figure}

\begin{figure}[h]
	\centering
	\includegraphics[width=0.8\textwidth]{images/tabular_processor.png}
	\caption[Tabular Processor Design]{Tabular Processor Design}
	\label{fig:tabular_processor}
\end{figure}

\Autoref{fig:tabular_processor} shows how the \textit{strategy design} pattern is realized.
The overall \textit{Strategy} class is defined in the \textit{TabularProcessor} class in the form of an abstract class with three core methods each tabular processing mechanism needs to implement.
These functions are the \textit{fit}, \textit{transform}, and \textit{inverse transform} functions (FR2.1 - FR2.3).

Each tabular processing strategy is different and realizes the transformation of the data in a different way by implementing the abstract functions of the parent strategy.
Therefore, new tabular processing mechanisms can be easily added by just implementing the methods of the abstract class (R2, FR2).
Furthermore, the tabular processing mechanism can be exchanged easily since they share the same functions.
This is handled in the \textit{TabularDataController} class, which is equivalent to the \textit{Context} of the \textit{strategy design} pattern.
The \textit{TabularDataController} handles all relevant aspects to use the tabular processing mechanisms, including but not limited to the instantiation, fitting, saving, and loading of \textit{TabularProcessor} instances.
The data, whose columns are segmented into categorical, numerical, and target as numpy arrays \cite{harris2020array}, needs to be provided to instantiate a tabular processor.
Target refers to the column of the dataset that a model tries to predict or estimate in a classification or regression scenario.
The first step in using the tabular processor is fitting it to the data.
Additional information required for the fitting process can be provided through meta-data in the form of a dictionary, which may change for different realizations.
Note that this fit function might not be necessary, depending on the processing mechanism, but it is required to be implemented anyways.
Only after fitting the transform function can be called transforming the data format and returning the transformed data in the same data segmentation structure, \ie separated into categorical, numerical and target arrays.
The inverse transformation works similarly, reversing transformed data back to its original format.
For clarity, please note the difference between the terms "data format" and "data segmentation" in this thesis.
The term "data segmentation" refers to the separation of columns into categorical, numerical, and target segments.
On the other hand, the term "data format" refers to the modification of the data values achieved through the encoding and decoding processes of the \textit{TabularProcessor}, specifically via their "transform" and "inverse transform" functions.

\subsection{Tabular Processor Implementations}
\label{ch:architecture-tabularProcessor-implementations}

Three different tabular processor versions have been implemented.
The \textit{IdentityProcessor} does not do anything to the data.
It is used to do experiments without any tabular processing mechanism and allows reproducing the original results of \cite{kotelnikov2022TabDDPMModellingTabular}.
In this thesis, only two approaches have been implemented, the \gls{bgm} and \gls{ft} tabular processors.
Other potential tabular processing mechanisms have been considered but usually lack one of the criteria specified in \Autoref{ch:Concept-criteria}.
The biggest challenge usually is the reversibility, since several encoding strategies only allow an encoding of the data but not a decoding of already encoded data.
The reason for this is that usually, decoding of encoded data is not necessary for most other deep learning tasks, where the encoded data is directly fed into the model.
Consequently, a decoding mechanism would need to be developed if one is not present, which is depending on the encoding technique, not a straightforward task and would likely require a special decoding model that is specifically trained for the encoding technique at hand.
However, this would exceed the scope of this thesis, as it would be very complex and time-consuming.
Therefore, it remains an open task for future researchers.

\subsubsection{BGMProcessor}
\label{ch:BGMProcessor}

The BGMProcessor\footnote{The name was chosen since a \acrfull{bgm} is the main component used in the processor.} is the same processing mechanism that has been introduced by the CTABGAN+ model \cite{zhao2022CTABGANEnhancingTabular}.
This processing mechanism was chosen since it fulfills all criteria mentioned in \Autoref{ch:Concept-criteria}.
Moreover, it is part of a state-of-the-art tabular data synthesis \gls{gan}-based model, which has already proven to be successful.

Firstly, a mixed-type encoding is introduced, specifically designed to encode single columns containing a mixed data type of numerical and categorical data types.
For the continuous values in the column, the authors adapt the mode-specific normalization technique from \cite{xu2019ModelingTabularData} using a \gls{bgm} implementation \cite{scikit-learndevelopers2023BayesianGaussianMixture}, which is a \gls{vgm} (see \Autoref{sec:dataNormalization} for details).
For the categorical values, the \gls{oh} vector $\beta$, which usually just indicates how many modes for normalization have been used, is extended by the number of possible categorical values in the column.
If a categorical value should be encoded, the $\alpha$ value is set to 0, and the \gls{oh} encoding indicating the specific category is set to 1.
Additionally, the authors allow missing values, by modeling them as another categorical entry.

Furthermore, the authors introduce a "general transform" mechanism \cite[p. 7]{zhao2022CTABGANEnhancingTabular} that is supposed to be used for simple distributions and counters the problem of exploding dimensionality caused by a \gls{oh} encoding \cite{zhao2022CTABGANEnhancingTabular}.
This general transformation is mapping the values into a range of [-1, 1], which is achieved through a shifted and scaled min-max normalization (see min-max scaler in \Autoref{sec:dataNormalization}).
Mathematically, the encoded datapoint $x^t_i$ can be computed using the original datapoint $x_i$ in column $x$ through the following formula:
\begin{equation}
	\begin{align*}
		x^t_i=2* \frac{x_i-\min(x)}{\max(x)-\min(x)}-1
	\end{align*}
\end{equation}
Which can easily be reversed in the following way:
\begin{equation}
	\begin{align*}
		X_i = (\max(x)-\min(x))*\frac{x^t_i+1}{2}+\min(x)
	\end{align*}
\end{equation}
Categorical values are label encoded before applying the normalization \cite{zhao2022CTABGANEnhancingTabular}.
The authors observe that this simplistic general transformation for continuous values only works well for simple distributions (\eg single-mode Gaussian) and not for complex distributions, for which the mode-specific normalization is preferred \cite{zhao2022CTABGANEnhancingTabular}.
For categorical values, the general transformation should only be applied if the number of categories a column can take is very high and would lead to very sparse vectors that make computations complex \cite{zhao2022CTABGANEnhancingTabular}.
Lastly, the authors log-transform columns that suffer from very long tails (see \Autoref{sec: synthetic tabular data generation}) because \Glspl{bgm} seem to have difficulties encoding values towards the tail \cite{zhao2022CTABGANEnhancingTabular}.
The log-transformation of each value $\tau$ in the column will be compressed to $\tau^c$, given a lower bound $l$ using:
\begin{equation}
	\label{eqn:log-transform}
	\begin{align*}
		\tau^c =
		\begin{cases}
			log(\tau)            & \text{if } l>0                                 \\
			log(\tau-l+\epsilon) & \text{if } l\leq0 \text{, where } \epsilon > 0
		\end{cases}
	\end{align*}
\end{equation}

This encoding strategy is realized in the BGMProcessor class, which uses the DataPrep and DataTransformer class, as developed by \cite{zhao2022CTABGANEnhancingTabular}.
These classes require additional information about the dataset, provided by the user, including what columns are categorical, categorical columns that have a high cardinality/dimensionality\footnote{Inside the software of the authors, they refer to it as "non\_categorical\_columns", which is a bit misleading. "Non\_categorical\_columns" are categorical columns with high dimensionality that are transformed into numerical columns and handled as if they were continuous.},
numerical, mixed (including what values are categorical), as well as what columns require a log transformation and which a general transformation.
This information is provided to the \textit{BGMProcessor} by extending the info.json (\Autoref{lst:info}) with an additional entry, \textit{dataset\_config}, see \Autoref{lst:info_extended} for an example.

\begin{lstlisting}[label={lst:info_extended},caption={Example extended data info file from the adult dataset (\Autoref{ch:methods-datasets})}]
    {
    "name": "Adult",
    [...]
    "val_size": 6513,
    "dataset_config": {
        "cat_columns": [ // categorical columns
            "workclass", 
            (...), 
            "income"
            ],
        "non_cat_columns": [], // high dim-categorical
        "log_columns": [], // log-transformation
        "general_columns": ["age"], //  general-transformation             
        "mixed_columns": { // mixed-data-types
            "capital-loss": [0.0], // the "0.0" value is categorical           
            "capital-gain": [0.0]
            },
        "int_columns": [ // numerical columns 
            "age", 
            (...), 
            "hours-per-week"
        ],
        "problem_type": "binclass", // [binclass|multiclass|regression]
        "target_column": "income"
        }
    }
\end{lstlisting}

\subsubsection{FTProcessor}
\label{ch:FTProcessor}

The \textit{FTProcessor}, short for \textit{Feature Tokenization Processor}, is based upon the work of \textcite{zheng2022DiffusionModelsMissing} and \textcite{gorishniy2021RevisitingDeepLearning}.
Similar to the \gls{bgm} variant, it fulfills all criteria stated in \Autoref{ch:Concept-criteria}.
However, the \gls{ft} mechanism has only been used in a data imputation task so far, where only individual cells must be synthesized.
It is unclear whether its functionality can be extended to multiple-cell synthesis.

The \textit{FTProcessor} transforms a tabular data input $x$ with $k$ features (or columns) into a static embedding representation $T \in \mathbb{R}^{k\times d}$ with $d$ as the embedding dimensionality.
Categorical columns are embedded through an embedding layer, which is basically a look-up table \cite{pytorch2023EmbeddingPyTorch13} of fixed size.
Each categorical input tensor of dimensions $[N]$ \\(\ie a table with $N$ columns) will be transformed in an embedding of size $[N,H]$ with $H$ as the embedding dimensionality \cite{gorishniy2021RevisitingDeepLearning}.
Numerical columns will be processed by a simple multiplication of linear layers weights $W$ \cite{gorishniy2021RevisitingDeepLearning}.
Lastly, a bias term $b$ is added to both categorical and numerical columns.
\Autoref{fig:ft} illustrates how a tabular data row with three numerical and two categorical features could be transformed into a new tensor \cite[Figure 2a, p.4]{gorishniy2021RevisitingDeepLearning}.

\begin{figure}[h]
	\centering
	\includegraphics[width=0.6\textwidth]{images/ft.png}
	\caption[Feature Tokenization]{Feature Tokenization \cite[Figure 2a, p.4]{gorishniy2021RevisitingDeepLearning}}
	\label{fig:ft}
\end{figure}

To revert the synthetic data that the diffusion model produces to the original data format, the numerical and categorical columns need to be handled differently.
The corresponding embedding weights divide the numerical output element-wise, and the average value is computed as the final output \cite{zheng2022DiffusionModelsMissing}.
This transforms output having the same dimensionality as the embedding ($d$) back into a single value \cite{zheng2022DiffusionModelsMissing}.
For categorical output, the closest categorical embedding is determined by computing the Euclidean distance between the produced output and every categorical embedding \cite{zheng2022DiffusionModelsMissing}.

In the original work of \cite{gorishniy2021RevisitingDeepLearning}, the feature tokenizer is directly applied in front of a transformer model, allowing the gradients to flow through the embedding and linear layers during training.
This enables the model to learn a meaningful embedding.
In the adaptation of \cite{zheng2023DiffusionModelsMissing} and in this thesis, the weights of the layers are frozen.
Hence, learning is not possible, and the embedding is static.
The reason for this in this thesis have been because of the overall software architecture.
The data transformation and the actual model training are fully separated,
resulting in gradients that do not flow back to the feature tokenizer to update any weights of the embeddings.
To achieve the update of weights, the tokenizer would need to be fully integrated into the diffusion model architecture, which in turn could have negatively affected the overall maintainability, compatibility, and performance (see R2-R4, \Autoref{ch:requirements}) of the software.

\subsection{Changes to the existing software}
\label{ch:methods-changes}
The changes made to the existing software (\Autoref{ch:conceptualDesign-existingCodeBase}) are various and include multiple major changes as well as a lot of small changes.
Below, the major changes to the individual scripts are listed.
An additional visualization of the process of the scripts can be found in Appendix \ref{A:activity_diagrams}

\begin{description}
	\item[train.py:]
		Before the dataset is preprocessed and transferred into a dataset class, the \textit{TabularDataController} handles the tabular processing mechanisms.
		For this, a \textit{TabularDataController} is instantiated at the beginning and either loads a previously fit \textit{TabularProcesser} instance or fits and saves the \textit{TabularProcesser} on the training data.
		It receives context information on which tabular processing mechanisms and which properties the dataset has through the configuration file and handles the
		instantiation, fitting, and transformation of the raw data.
		Despite the addition of the \textit{TabularDataController}, the actual training remained untouched.
		Hence, introducing a \textit{TabularProcesser} only changes the data format (and maintains the segmented data structure) the diffusion model will receive and does not change the actual training loop.

	\item[sample.py:]
		A TabularDataController is instantiated at the beginning and loads a previously fit \textit{TabularProcesser} instance.
		If there is no previous \textit{TabularProcesser} instance to be loaded, it will call the fit function again.
		After the sampling is finished, the \textit{TabularProcesser}'s inverse transform function is called to bring the synthetic data into the original data format.

	\item[eval\_similarity.py:]
		This script does not replace the previous evaluation script.
		Instead, it is called after the original evaluation script (see \textit{pipeline.py}).
		First, three \textit{TabularDataController} instances are created, one for each train-, validation- and test data split.
		With the \textit{TabularDataController} instances, \gls{pd} \cite{mckinney-proc-scipy-2010} data frames are created, which are required for the TabSynDex similarity score.
		After the TabSynDex score is calculated, visualizations are created, if set by the user.
	\newpage
	\item[pipeline\_*.py:]
		The basic pipeline hardly changed.
		The only major change is that the similarity score evaluation is performed after the default evaluation using machine learning efficacy.
		In this evaluation, the TabSynDex metric is calculated as well as (if set by the user) visualization are created.

	\item[tune\_*.py:]
		Inside the tuning script, the objective function changed.
		If set by the user, the objective function will be based on the TabSynDex similarity score instead of the machine learning efficacy.
		To reduce processing time, visualizations inside the similarity evaluation part will, by default, only be created after the hyperparameters have been optimized and the eval\_seeds script will be executed.

	\item[eval\_seeds.py:]
		The basic eval\_seeds script did not change a lot.
		Again, the TabSynDex metric will be called after the default machine learning efficacy.
\end{description}

In order to easily change the \textit{TabularProcesser} type that should be executed in an experiment, the experiment configuration displayed as \Autoref{lst:configuration} was extended.
The additional configuration section \textit{tabular\_processor} has been added, containing a \textit{type} key.
The \textit{type} key can be either set to "identity", "ft" or "bgm" to indicate, which \textit{TabularProcesser} type should be used.

\section{Experimental Setup}
\label{ch:methods-experimentalSetup}

\subsection{Experiments}
\label{ch:Experiments}

To systematically analyze the performance of the different tabular processing mechanisms, the following baseline experiments have been performed, creating a set of baseline models:

\begin{description}
	\item[Baseline-Real:] Comparison of the real training data with the real test data.
		The similarity evaluation is expected to be very high since the data is from the same joint distribution.
		The computed machine learning efficacy score can be seen as an objective value that should be reached.
		The closer the synthetic data based machine learning efficacy results are to this target, the better the synthetic data.
	\item[Baseline-TabDDPM:] Reproduction of the TabDDPM results of the authors \cite{kotelnikov2022TabDDPMModellingTabular} but with the extend evaluation explained in \Autoref{ch:conceptualDesign-Evaluation} (Machine learning efficacy + TabSynDex + Visualizations).
	\item[Baseline-TVAE:] Like Baseline-TabDDPM but for TVAE.
	\item[Baseline-CTABGAN:] Like Baseline-TabDDPM but for CTABGAN.
	\item[Baseline-CTABGAN+:] Like Baseline-TabDDPM but for CTABGAN+.
	\item[Baseline-SMOTE:] Like Baseline-TabDDPM but for SMOTE.
\end{description}

\noindent For each implemented Tabular Processing mechanism, the following three experiments should be performed:

\begin{description}
	\item[Experiment 1:] Tabular Processing combined with TabDDPM with additional similarity evaluation. Hyperparameters tuned like in the original experiment (tuned after CatBoost-machine learning efficacy score).
	\item[Experiment 2:] Like Experiment 1, but hyperparameters are optimized towards the TabSynDex similarity score instead of the machine learning efficacy score.
	\item[Experiment 3:] Like Experiment 2, but the preprocessing strategy, that normalizes numerical data after the tabular processing transformation is replaced from a quantile transform function to a min-max transformation or is completely removed.
\end{description}

\noindent Depending on the results of the first experiments, subsequent experiments may only be performed on a subset of promising Tabular processing mechanisms.
To not get confused with the different model versions, the models are named after the following principal:
\begin{quote}
Modelname[-TabularProcessor]$^{tuning}_{preprocessing}$
\end{quote}
The Modelname is followed by an optional tabular processing mechanism.
The tuning strategy ("ml" for "machine learning efficacy score" or "s" for "TabSynDex similarity score") is indicated via a superscript (Experiment 2).
The preprocessing strategy ("q" for "quantile transform" or "m" for "min-max", or "n" for "none"/"no transformation") is indicated via a subscript (Experiment 3).

The min-max strategy was chosen as a commonly used linear transformation counterpart to the non-linear quantile transformation.
Since some tabular processing strategies already normalize or standardize the data, additional quantile or min-max transformations may not be required, 
which is why experiment 3 will also investigate the effect of applying no transformation after the tabular processor.

\subsection{Execution Environment}
\label{ch:environment}

The code was developed in python version 3.9.7 and made use of several libraries.
Most important libraries include (for a full list, please see the \textit{environment.yml} file in the source code):
\begin{itemize}
	\item azure-core, azureml-core: For running the code in the Microsoft Azure cloud \cite{microsoft2023CloudComputingServices}.
	\item catboost (1.0.3): contains the CatBoost model for the machine learning efficacy.
	\item pytorch (1.10.1): neural network framework.
	\item table\_evaluator (1.4.2): used to produce visualizations.
	\item numpy (1.21.4): allows efficient array computation.
	\item optuna (2.10.1): used for hyperparameter tuning.
	\item pandas (1.5.2): allows fast computations with dataframes of tabular data.
	\item scikit-learn (1.0.2): includes several utility functions for machine learning, \eg metrics such as accuracy or F1-score.
	\item TabSynDex \cite{chundawat2022UniversalMetricRobusta}: TabSynDex metric source code.
\end{itemize}

The training was performed using Microsoft Azure Machine Learning Studio.
Each experiment was executed on a STANDARD\_NC6 compute cluster running on a Linux distribution, consisting of six virtual \glspl{cpu} (Intel Xeon E5-2690v3) with 56 GB Memory, 340 GB (SSD) Storage and the computing of one-half \gls{gpu} (Tesla K80) with 12 GB \gls{gpu} memory \cite{vikancha-msft2022NCseriesAzureVirtual}.
All experiments could also be performed locally on a Windows machine using a \gls{cpu}.
Local experiments with a \gls{gpu} should be possible but have not been tested.

\subsection{Hyperparameters}
The hyperparameter search space for the various models was not changed and is equal to the experiments in \cite{kotelnikov2022TabDDPMModellingTabular}.
The search space for the different models is listed in \Autoref{tab:catboost_tune,tab:diff_tune,tab:tvae_tune,tab:ctabgan_tune}. 
\begin{table}[h]
	\centering
	\begin{tabular}{lr}
		\toprule
		Parameter               & Distribution                       \\
		\midrule
		Learning Rate           & LogUniform[1e-5,3e-3]              \\
		Batch Size              & Cat\{256,4096\}                    \\
		Diffusion timesteps     & Cat\{100,1000\}                    \\
		Training iterations     & Cat\{5000,10000,20000\}            \\
		\# MLP layers           & Int\{2,4,6,8\}                     \\
		MLP layer width         & Int\{128,256,512,1024\}            \\
		Proportion of samples   & Float\{0.25, 0.5, 1, 2\}           \\
		\midrule
		Train size              & \#entries in training dataset      \\
		\# Samples              & Proportion of samples * Train size \\
		Dropout                 & 0.0                                \\
		Scheduler               & cosine                             \\
		Gaussian diffusion loss & mse                                \\
		\midrule
		Number of tuning trials & 100                                \\
		\bottomrule
		\multicolumn{2}{c}{}\\[-0.6em]
	\end{tabular}
	\caption[TabDDPM Hyperparameter Search Space]{TabDDPM model hyperparameter tuning search space \cite{kotelnikov2022TabDDPMModellingTabular}}
	\label{tab:diff_tune}
\end{table}
\newpage

\begin{table}[H]
	\centering
	\begin{tabular}{lr}
		\toprule
		Parameter               & Distribution                     \\
		\midrule
		\# classif. layers      & UniformInt[1,6]                  \\
		Classif. layer size     & Int\{62, 128, 256, 512\}         \\
		Training iterations    & Cat\{5000, 20000, 30000\}        \\
		Batch size              & Cat\{456,4096\}                  \\
		Embedding dim.          & Int\{16,32,64,128,256,512,1024\} \\
		Loss factor             & LogUniform[0.01, 10]             \\
		Proportion of samples   & Float\{0.25, 0.5, 1, 2, 4, 8\}   \\
		\midrule
		Number of tuning trials & 100                              \\
		\bottomrule
		\multicolumn{2}{c}{}\\[-0.6em]
	\end{tabular}
	\caption[TVAE Hyperparameter Search Space]{TVAE model hyperparameter tuning search space \cite{kotelnikov2022TabDDPMModellingTabular}}
	\label{tab:tvae_tune}
\end{table}

\begin{table}[H]
	\centering
	\begin{tabular}{lr}
		\toprule
		Parameter               & Distribution                   \\
		\midrule
		\# classif. layers      & UniformInt[1,4]                \\
		Classif. layer size     & Int\{62, 128, 256\}            \\
		Training iterations    & Cat\{1000, 5000, 7500\}        \\
		Batch size              & Int\{512,1024,2048\}           \\
		random dim.             & Int\{16,32,64,128\}            \\
		\# Channels             & Int\{16, 32, 64\}              \\
		Proportion of samples   & Float\{0.25, 0.5, 1, 2, 4, 8\} \\
		\midrule
		Number of tuning trials & 30                             \\
		\bottomrule
		\multicolumn{2}{c}{}\\[-0.6em]
		\multicolumn{2}{c}{}\\[-0.6em] % Testen
	\end{tabular}
	\caption[CTABGAN(+) Hyperparameter Search Space]{CTABGAN/CTABGAN+ model hyperparameter tuning search space. Training iterations and the number of tuning trails were reduced compared to the original \cite{kotelnikov2022TabDDPMModellingTabular}, to reduce computation time.}
	\label{tab:ctabgan_tune}
\end{table}

\begin{table}[H]
	\centering
	\begin{tabular}{lr}
		\toprule
		Parameter                 & Distribution        \\
		\midrule
		Max depth                 & UniformInt[3, 10]   \\
		Learning rate             & LogUniform[1e-5, 1] \\
		Bagging temperature       & Uniform[0,1]        \\
		L2 leaf reg               & LogUniform[1,10]    \\
		Leaf estimation iteration & UniformInt[1,10]    \\
		\midrule
		Number of tuning trials   & 100                 \\
		\bottomrule
		\multicolumn{2}{c}{}\\[-0.6em]
	\end{tabular}
	\caption[CatBoost Hyperparameter Search Space]{CatBoost evaluation model hyperparameter tuning search space (proposed by \cite{gorishniy2021RevisitingDeepLearning})}
	\label{tab:catboost_tune}
\end{table}

\section{Dataset}
\label{ch:methods-datasets}

The dataset that was used for the experiments was the \textit{Adult} dataset, also known as "Census Income" from the UCI Machine Learning repository \cite{Dua:2019}.
The dataset was constructed through an extraction from a 1994 census database \cite{kohavi1996ScalingAccuracyNaiveBayes}.
Overall, the dataset has 15 columns, of which nine are categorical and six are numerical.
In total, the dataset consists of 48842 rows.\newline

The numerical columns are: \textit{age}, \textit{fnlwgt}\footnote{Short for "final weight", a sampling weight.}, \textit{education-num}, \textit{capital-gain}, \textit{capital-loss} and \textit{hours-per-week}.

The categorical columns are: \textit{workclass}, \textit{education}, \textit{marital-status}, \textit{occupation}, \textit{relationship}, \textit{race}, \textit{sex}, \textit{native-country}, and \textit{income} (target column). \newline


The dataset was created for binary classification, where the model should predict the column value of the income column (either \textit{>50K} or \textit{<=50K}).
\Autoref{tab:adult} shows five example entries of the adult dataset:

\begin{table}[h]
	\centering
    \begin{subtable}{\textwidth}
        \centering
        \resizebox{\columnwidth}{!}{
            \begin{tabular}{|c|c|c|c|c|c|c|c|}
                \toprule
                \textbf{age} & \textbf{workclass} & \textbf{fnlwgt} & \textbf{education} & \textbf{education-num} & \textbf{marital-status} & \textbf{occupation} & \textbf{relationship} \\
                \midrule
                39.0         & State-gov          & 77516.0         & Bachelors          & 13.0                   & Never-married           & Adm-clerical        & Not-in-family         \\
                50.0         & Self-emp-not-inc   & 83311.0         & Bachelors          & 13.0                   & Married-civ-spouse      & Exec-managerial     & Husband               \\
                38.0         & Private            & 215646.0        & HS-grad            & 9.0                    & Divorced                & Handlers-cleaners   & Not-in-family         \\
                53.0         & Private            & 234721.0        & 11th               & 7.0                    & Married-civ-spouse      & Handlers-cleaners   & Husband               \\
                28.0         & Private            & 338409.0        & Bachelors          & 13.0                   & Married-civ-spouse      & Prof-specialty      & Wife                  \\
                \bottomrule
				\multicolumn{7}{c}{}\\[-0.6em]
            \end{tabular}
        }
        \caption{First eight columns of Adult income dataset}
        \label{subtab:adult1}
    \end{subtable}
    \par\bigskip % add some space between the two tables
    \begin{subtable}{\textwidth}
        \centering
        \resizebox{\columnwidth}{!}{
            \begin{tabular}{|c|c|c|c|c|c|c|}
                \toprule
                \textbf{race} & \textbf{sex} & \textbf{capital-gain} & \textbf{capital-loss} & \textbf{hours-per-week} & \textbf{native-country} & \textbf{income} \\
                \midrule
                White         & Male         & 2174.0                & 0.0                   & 40.0                    & United-States           & $\leq$50K       \\
                White         & Male         & 0.0                   & 0.0                   & 13.0                    & United-States           & $\leq$50K       \\
                White         & Male         & 0.0                   & 0.0                   & 40.0                    & United-States           & $\leq$50K       \\
                Black         & Male         & 0.0                   & 0.0                   & 40.0                    & United-States           & $\leq$50K       \\
                Black         & Female       & 0.0                   & 0.0                   & 40.0                    & Cuba                    & $\leq$50K       \\
                \bottomrule
				\multicolumn{7}{c}{}\\[-0.6em]
            \end{tabular}
        }
        \caption{Second seven columns of Adult income dataset, including the dataset target column "income"}
        \label{subtab:adult2}
    \end{subtable}
	\caption[Example Adult Dataset]{Adult income dataset with five exemplary entries}
	\label{tab:adult}
\end{table}



\chapter{Results}
\label{ch:results}

\section{Reproduction and Verification of Results}
\label{ch:results-reproduction}

Given that this thesis draws heavily upon the research conducted by \cite{kotelnikov2022TabDDPMModellingTabular},
it was deemed absolutely necessary to first reproduce the original experiments and subsequently verify the authors findings,
prior to conducting any new experiments.
Fortunately, the publicly available code \cite{akim2023TabDDPMModellingTabular}, facilitated the replication of the experiments with relative ease.

The reproduction is limited to the adult dataset from \autoref{ch:methods-datasets} and to the machine learning efficacy computed with regards to a tuned CatBoost \cite{prokhorenkova2018CatBoostUnbiasedBoosting} model.

Firstly, a CatBoost model was tuned on the adult dataset, using the provided tuning script (tune\_evaluation\_model.py) [TODO: Drin lassen oder weg lassen?].
Next, for each sampling algorithm, a model was trained, according to the configuration files provided by the authors.
Each configuration file contains the parameters the authors found during hyperparameter tuning.
Hence, with the models respective pipeline.py script, the best found model from hyperparameter tuning could be trained and saved.
Finally, the trained CatBoost and trained sampling model were used in the evaluation script (eval\_seeds.py), which calculates and reports the results.

\begin{table}[h]
	\centering
	\begin{tabular}{l|c|c|r}
		\hline
		\textbf{Model}     & \textbf{Reproduction} & \textbf{Original} & \textbf{Difference} \\ \hline
		Real               & 0.815                 & 0.815             & 0                   \\ \hline
		TVAE$^{ml}$        & 0.780                 & 0.781             & -0.001              \\ \hline
		CTABGAN$^{ml}$     & 0.775                 & 0.783             & -0.008              \\ \hline
		CTABGAN+$^{ml}$    & 0.775                 & 0.772             & +0.003              \\ \hline
		SMOTE$^{ml}$       & 0.791                 & 0.791             & 0                   \\ \hline
		TabDDPM$^{ml}_{q}$ & 0.794                 & 0.795             & -0.001              \\ \hline
	\end{tabular}
	\caption[Reproduction of original Results]{Comparison of the CatBoost F1-score on synthetic datasets, created by different sampling models.
		F1-Scores of the reproduction experiments are compared against the results reported by the original authors \cite[Table 4, p. 8]{kotelnikov2022TabDDPMModellingTabular}.}
	\label{tab:reproduction}
\end{table}

\autoref{tab:reproduction} shows the computed F1-scores achieved by the CatBoost model when trained on different synthetic datasets generated by different sampling models.
The scores that could be reproduced are almost exactly the same as the scores reported by the original authors.
All differences are within the standard deviation reported by the authors, except for the CTABGAN-model.
It is unclear, why the CTABGAN-model score deviates in the reproduction experiment from the original score.
It is important to note, that minor modifications to the code or different python library versions have caused this alternation, which where required in order to train the model in the cloud environment, specified in \autoref{ch:methods-experimentalSetup}.

Therefore, the results as reported by the authors could be overall reproduced and verified.

\section{Metric Results}
\label{ch:results-Metric-results}

% QUALITATIVE vs QUANTITIVE

\subsection{Baseline Experiments}
\label{ch:Baseline}

After verification that the models are able to reproduce the machine-learning efficacy scores as reported,
their performance is additionally evaluated using the similarity evaluation as proposed in \autoref{ch:conceptualDesign-Evaluation}.
\autoref{tab:ml_baseline} shows the complete machine learning efficacy score results for the different sampling techniques:

\begin{table}[h]
	\centering
	\begin{tabular}{l|c|c|c}
		\hline
		\textbf{Model}     & \textbf{Accurarcy} & \textbf{F1}    & \textbf{ROC-AUC} \\ \hline
		Real               & 0.874              & 0.815          & 0.928            \\ \hline
		TVAE$^{ml}$        & 0.845              & 0.781          & 0.900            \\ \hline
		CTABGAN$^{ml}$     & 0.850              & 0.775          & 0.900            \\ \hline
		CTABGAN+$^{ml}$    & 0.855              & 0.775          & 0.907            \\ \hline
		SMOTE$^{ml}$       & 0.858              & 0.791          & 0.910            \\ \hline
		TabDDPM$^{ml}_{q}$ & \textbf{0.860}     & \textbf{0.794} & \textbf{0.913}   \\ \hline
	\end{tabular}
	\caption[Machine learning efficacy baseline]{Machine learning efficacy (CatBoost) baseline results.}
	\label{tab:ml_baseline}
\end{table}


In addition to the machine learning efficacy scores, the similarity scores of the TabSynDex metric are computed (\autoref{tab:sim_baseline}).

\begin{table}[h]
	\centering
	\begin{tabular}{lrrrrrr}
		\toprule
		\textbf{Model}     & \textbf{Similarity Score} & \textbf{Basic} & \textbf{Correlation} & \textbf{ML}    & \textbf{Support} & \textbf{pMSE}  \\
		\midrule
		Real               & 0.960                     & 0.992          & 0.943                & 0.998          & 0.984            & 0.882          \\
		TVAE$^{ml}$        & 0.658                     & 0.854          & 0.814                & 0.962          & 0.657            & 0.000          \\
		CTABGAN$^{ml}$     & 0.741                     & 0.940          & 0.832                & 0.984          & \textbf{0.947}   & 0.000          \\
		CTABGAN+$^{ml}$    & 0.750                     & 0.969          & 0.882                & 0.990          & 0.892            & 0.019          \\
		SMOTE$^{ml}$       & 0.723                     & 0.953          & 0.865                & \textbf{0.992} & 0.804            & 0.000          \\
		TabDDPM$^{ml}_{q}$ & \textbf{0.759}            & \textbf{0.973} & \textbf{0.919}       & \textbf{0.992} & 0.874            & \textbf{0.035} \\
		\bottomrule
	\end{tabular}
	\caption[Similarity baseline]{Similarity baseline results using the TabSynDex metrices. Similarity Score is the average of the other five scores. Best result is highlighted in bold}
	\label{tab:sim_baseline}
\end{table}

These baseline experiments show, that the Diffusion based synthesis approach outperforms other models not only in terms of machine learning efficacy, but also in terms of other metrics.
However, \autoref{tab:sim_baseline} shows that the CTABGAN models outperform TabDDPM in the Support coverage metric.
Additionally, it is worth mentioning that even though TabDDPM achieves the highest \gls{pmse} score, it is still extremely low and almost 0, which is the same for all other models.
The authors of \cite{chundawat2022UniversalMetricRobust} essentially confirm this observation.
During their experiments, the tested sampling techniques (various \gls{gan}-based approaches) also struggle to produce any synthetic data which achieves a \gls{pmse} score that is noticeably higher than 0.
\autoref{tab:sim_baseline} indicates, that this observation also holds for a diffusion based approach, which hyperparameters were tuned towards a machine learning efficacy score using a CatBoost model.

\subsection{Experiment 1: Adding Tabular Processing}
\label{ch:Experiment-1}

In the first set of experiments, the different tabular processing mechanism described in \autoref{ch:architecture-tabularProcessor-implementations} are evaluated.
For this, the tabular processing mechanisms have been added to the pipeline of TabDDPM, as described in the concept in \autoref{fig:Overall_changed}.
Consequently, the tuning (tune\_ddpm.py, see \autoref{ch:scripts}) of the diffusion model with the additional tabular processing was required.
The models hyperparameter have again been tuned towards the machine learning efficacy of a CatBoost model.

The results of the machine learning efficacy and TabSynDex metric results can be found in \autoref{tab:exp1-ml} and \autoref{tab:exp1-sim} respectively.
\begin{table}[h]
	\centering
	\begin{tabular}{lrrr}
		\toprule
		\textbf{Model}         & \textbf{Accurarcy} & \textbf{F1}    & \textbf{ROC-AUC} \\
		\midrule
		Real                   & 0.874              & 0.815          & 0.928            \\
		TabDDPM$^{ml}_{q}$     & 0.860              & 0.794          & 0.913            \\
		TabDDPM-BGM$^{ml}_{q}$ & \textbf{0.863}     & \textbf{0.798} & \textbf{0.916}   \\
		TabDDPM-FT$^{ml}_{q}$  & 0.785              & 0.552          & 0.821            \\
		\bottomrule
	\end{tabular}
	\caption[Experiment1-ML-Efficacy]{CatBoost Machine learning efficacy scores for different tabular processing techniques.}
	\label{tab:exp1-ml}
\end{table}

\begin{table}[h]
	\centering
	\begin{tabular}{lrrrrrr}
		\toprule
		\textbf{Model}         & \textbf{Similarity Score} & \textbf{Basic} & \textbf{Correlation} & \textbf{ML}    & \textbf{Support} & \textbf{pMSE}  \\
		\midrule
		Real                   & 0.960                     & 0.992          & 0.943                & 0.998          & 0.984            & 0.882          \\
		TabDDPM$^{ml}_{q}$     & \textbf{0.759}            & \textbf{0.973} & \textbf{0.919}       & 0.992          & 0.874            & \textbf{0.035} \\
		TabDDPM-BGM$^{ml}_{q}$ & 0.742                     & 0.964          & 0.918                & \textbf{0.996} & 0.831            & 0.000          \\
		TabDDPM-FT$^{ml}_{q}$  & 0.595                     & 0.495          & 0.648                & 0.869          & \textbf{0.963}   & 0.000          \\
		\bottomrule
	\end{tabular}
	\caption[Experiment1-Similarity]{TabSynDex evaluation metric scores for different tabular processing techniques.}
	\label{tab:exp1-sim}
\end{table}

Both evaluations show, that the additional \gls{bgm} tabular processing seems to increase the ML-efficacy scores.
All metrics in \autoref{tab:exp1-ml} are highest for the TabDDPM-BGM model and the ML efficacy score of TabSynDex (which makes use of different models)
is highest for TabDDPM-BGM as well, although only by a slight margin.
\autoref{tab:exp1-sim} indicates, that this increase seems to come at the cost of reduced performance in the other metrices, which are highest for the Basic-, Correlation- and pMSE-Score for the plain TabDDPM version.
TabDDPM-FT compares significantly worse than its counterparts in the Correlation and Basic similarity score.
Interestingly, TabDDPM-FT performance significantly better in the Support score than the other versions and approximately 13 percentage-points worse in terms of ML efficacy computed by the TabSynDex metric.
More details can be seen in the \autoref{tab:exp1-ml}, that shows that especially the F1 score from TabDDPM-FT is much worse than the F1 score of the other models.
Lastly, neither the \gls{bgm} nor the \gls{ft} tabular processing enable the diffusion model to produce synthetic data that is able to increase the \gls{pmse} score.


\subsection{Experiment 2: Similarity Hyperparameter optimization}
\label{ch:Experiment-2}

The second set of experiments are very similar to the first experiments.
Instead of tuning the models hyperparameters after the machine learning efficacy, as proposed by the original authors,
the models hyperparameters are tuned after the TabSynDex similarity score.

The results of the machine learning efficacy and TabSynDex metric results can be found in \autoref{tab:exp2-ml} and \autoref{tab:exp2-sim} respectively.

TODO FT simTune; ctabgan+simTUne

\begin{table}[h]
	\centering
	\begin{tabular}{lrrr}
		\toprule
		\textbf{Model}        & \textbf{Accurarcy} & \textbf{F1}    & \textbf{ROC-AUC} \\
		\midrule
		Real                  & 0.874              & 0.815          & 0.928            \\
		TabDDPM$^{s}_{q}$     & 0.856              & 0.782          & 0.908            \\
		TabDDPM-BGM$^{s}_{q}$ & \textbf{0.859}     & \textbf{0.792} & \textbf{0.911}   \\
		TabDDPM-FT$^{s}_{q}$  & 0.767              & 0.450          & 0.712            \\
		CTABGAN$^{s}$         & 0.850              & 0.776          & 0.900            \\
		TVAE$^{s}$            & 0.845              & 0.780          & 0.900            \\
		\bottomrule
	\end{tabular}
	\caption[Experiment2-ML-Efficacy]{CatBoost Machine learning efficacy scores for different tabular processing techniques which hyperparameter have been tuned towards the TabSynDex similarity score.}
	\label{tab:exp2-ml}
\end{table}

\begin{table}[h]
	\centering
	\begin{tabular}{lrrrrrr}
		\toprule
		\textbf{Model}        & \textbf{Similarity Score} & \textbf{Basic} & \textbf{Correlation} & \textbf{ML}    & \textbf{Support} & \textbf{pMSE}  \\
		\midrule
		Real                  & 0.960                     & 0.992          & 0.943                & 0.998          & 0.984            & 0.882          \\
		TabDDPM$^{s}_{q}$     & 0.852                     & 0.976          & \textbf{0.921}       & \textbf{0.991} & 0.952            & 0.420          \\
		TabDDPM-BGM$^{s}_{q}$ & \textbf{0.857}            & \textbf{0.982} & 0.858                & \textbf{0.991} & 0.920            & \textbf{0.532} \\
		TabDDPM-FT$^{s}_{q}$  & 0.589                     & 0.513          & 0.620                & 0.819          & \textbf{0.992}   & 0.000          \\
		CTABGAN$^{s}$         & 0.740                     & 0.938          & 0.833                & 0.984          & 0.947            & 0.000          \\
		TVAE$^{s}$            & 0.658                     & 0.856          & 0.815                & 0.962          & 0.656            & 0.000          \\
		\bottomrule
	\end{tabular}
	\caption[Experiment2-Similarity]{TabSynDex evaluation metric scores for different tabular processing techniques which hyperparameter have been tuned towards the TabSynDex similarity score.}
	\label{tab:exp2-sim}
\end{table}

The results show, that TabDDPM-BGM outperforms all other models in terms of the CatBoost machine learning efficacy and is on pair with TabDDPM in the TabSynDex ML-efficacy score.
TabDDPM-BGM also has the highest overall Similarity Score and achieves the highest Basic score and \gls{pmse} score.
The simpler TabDDPM however outperforms its \gls{bgm} counterpart in terms of Correlation and Support score.
Overall, TabDDPM achieves comparable performance to TabDDPM-BGM on all metrics, with the biggest difference in the \gls{pmse} score of - 11 percentage points.
TabDDPM-FT performance significantly worse than all other TabDDPM variants in all metrics except the support score.
On the one hand, it seems to be the case that hyperparameter tuning after the similarity score does have a big influence on the TabDDPM and TabDDPM-BGM models \gls{pmse} score.
On the other hand, this hyperparameter tuning did not affect the \gls{pmse} score of the other tested model, the CTABGAN, TVAE or the TabDDPM-FT.

\subsection[]{Experiment 3: Exchanging the Normalization}
\label{ch:Experiment-3}

So far, all TabDDPM models received data, that was normalize according to the quantile-transform function (applied after tabular processing), as proposed in \cite{kotelnikov2022TabDDPMModellingTabular}.
In a third experiment, it will be investigated how the models perform, when their quantile-transform function is replaced by a more simple MinMax-scaler, as explained in \autoref{sec:dataNormalization}.
Based upon the performance of the model variants so far, only the plain TabDDPM and TabDDPM-BGM, tuned after the similarity score have been investigated, since the performance of TabDDPM-FT was noticeably worse.


\begin{table}[h]
	\centering
	\begin{tabular}{lrrr}
		\toprule
		\textbf{Model}        & \textbf{Accurarcy} & \textbf{F1}    & \textbf{ROC-AUC} \\
		\midrule
		Real                  & 0.874              & 0.815          & 0.928            \\
		TabDDPM$^{s}_{m}$     & 0.856              & 0.778          & \textbf{0.910}   \\
		TabDDPM-BGM$^{s}_{m}$ & \textbf{0.857}     & \textbf{0.787} & 0.909            \\
		\bottomrule
	\end{tabular}
	\caption[Experiment3-ML-Efficacy]{CatBoost Machine learning efficacy scores for different tabular processing techniques which hyperparameter have been tuned towards the TabSynDex similarity score
		and the data was transformed according to a MinMax-scaler.}
	\label{tab:exp3-ml}
\end{table}

\begin{table}[h]
	\centering
	\begin{tabular}{lrrrrrr}
		\toprule
		\textbf{Model}        & \textbf{Similarity Score} & \textbf{Basic} & \textbf{Correlation} & \textbf{ML}    & \textbf{Support} & \textbf{pMSE}  \\
		\midrule
		Real                  & 0.960                     & 0.992          & 0.943                & 0.998          & 0.984            & 0.882          \\
		TabDDPM$^{s}_{m}$     & \textbf{0.869}            & 0.938          & \textbf{0.930}       & 0.990          & \textbf{0.928}   & \textbf{0.558} \\
		TabDDPM-BGM$^{s}_{m}$ & 0.856                     & \textbf{0.981} & 0.913                & \textbf{0.992} & 0.915            & 0.476          \\
		\bottomrule
	\end{tabular}
	\caption[Experiment3-Similarity]{TabSynDex evaluation metric scores for different tabular processing techniques which hyperparameter have been tuned towards the TabSynDex similarity score
		and the data was transformed according to a MinMax-scaler.}
	\label{tab:exp3-sim}
\end{table}

Given the results in \autoref{tab:exp3-ml} and \autoref{tab:exp3-sim} it seems to be the case that TabDDPM-BGM outperforms TabDDPM in terms of ML-efficacy and the Basic score, but only by a slight margin.
TabDDPM, on the other hand, achieves a higher overall similarity score, mainly due to the superior performance in the Correlation, Support and pMSE-score.


\section{Visual Results}
\label{ch:results-Visual}

Several plots have been produced accroding to the methodology of \cite{brenninkmeijer2019GenerationEvaluationTabular}.
This section will cover which plots have been produced and how the plots for different model versions are different.

\subsection[]{Correlation Difference Matrix}

To get an intuition on how well correlations within the dataset are reproduced, correlation matrices have been computed.
The correlation matrix for the synthetic dataset has been subtracted from the correlation matrix of the real dataset in order to produce a correlation difference matrix.
The smaller the difference, the more similar are the correlations within the synthetic dataset to the correlations within the real dataset.

Looking at the correlations matrix differences in \autoref{fig:corr_base} for the baseline models that are not diffusion based, one can see that the CTABGAN+$^{ml}$ and SMOTE approaches have the smallest
correlation matrix difference.
However, the diffusion model TabDDPM$^{ml}$ produces the matrix with the smallest overall differences.

\begin{figure}[h]
	\centering
	\begin{subfigure}{0.3\textwidth}
		\includegraphics[width=\textwidth]{images/correlation_difference/real.jpg}
		\caption{Real}

	\end{subfigure}
	\hfill
	\begin{subfigure}{0.3\textwidth}
		\includegraphics[width=\textwidth]{images/correlation_difference/tvae.jpg}
		\caption{TVAE$^{ml}$}

	\end{subfigure}
	\hfill
	\begin{subfigure}{0.3\textwidth}
		\includegraphics[width=\textwidth]{images/correlation_difference/ctabgan.jpg}
		\caption{CTABGAN$^{ml}$}
	\end{subfigure}

	\begin{subfigure}{0.3\textwidth}
		\includegraphics[width=\textwidth]{images/correlation_difference/ctabgan+.jpg}
		\caption{CTABGAN+$^{ml}$}

	\end{subfigure}
	\begin{subfigure}{0.3\textwidth}
		\includegraphics[width=\textwidth]{images/correlation_difference/smote.jpg}
		\caption{SMOTE}

	\end{subfigure}
	\begin{subfigure}{0.3\textwidth}
		\includegraphics[width=\textwidth]{images/correlation_difference/tab-ddpm.jpg}
		\caption{TabDDPM$^{ml}_q$}

	\end{subfigure}
	\caption{Correlation Matrix difference for Baseline models.}
	\label{fig:corr_base}
\end{figure}


\begin{figure}[h]
	\begin{subfigure}{0.3\textwidth}
		\includegraphics[width=\textwidth]{images/correlation_difference/tab-ddpm-bgm.jpg}
		\caption{TabDDPM-BGM$^{ml}_q$}
	\end{subfigure}
	\hfill
	\begin{subfigure}{0.3\textwidth}
		\includegraphics[width=\textwidth]{images/correlation_difference/tab-ddpm-ft.jpg}
		\caption{TabDDPM-FT$^{ml}_q$}
	\end{subfigure}
	\hfill
	\begin{subfigure}{0.3\textwidth}
		\includegraphics[width=\textwidth]{images/correlation_difference/tab-ddpm-bgm-simTune-minmax.jpg}
		\caption{TabDDPM-BGM$^{s}_m$}
	\end{subfigure}
	\caption{Correlation Matrix difference from selected experiment models.}
	\label{fig:corr_diffusion}
\end{figure}



\autoref{fig:corr_diffusion} shows that while the TabDDPM-BGM$^{ml}$ are able to produce small correlation differences, the TabDDPM-FT$^{ml}$ is not able to reproduce
the correlation matrix of the real dataset.
The overall best correlation difference matrix is produced by TabDDPM$^{s}_m$, which is also reflected in their TabSynDex correlation score in \autoref{tab:exp3-sim}, which is highest across all model versions.
Other model versions matrices do not show any noteworthy changes and are displayed in \autoref{fig_a:corr_diff}.

\subsection[]{Principle Component Analysis}
\label{ch:results-pca}

A \gls{pca} is a dimensionality reduction technique, that converts a dataset onto principal components \cite{brenninkmeijer2019GenerationEvaluationTabular}.
The components are then sorted according to the amount of variance from the original dataset each component is able to capture.
The following visualizations map the first two principal components, \ie the components that capture the most variance from data.


\begin{figure}[h]
	\centering
	\includegraphics[width=0.6\textwidth]{images/pca/pca.png}
	\caption{\gls{pca} for the real training and testing dataset}
	\label{fig:pca}
\end{figure}

\autoref{fig:pca} shows the two \gls{pca} plots that display the fist two principle components from the real training and testing dataset.
One can see that the majority of values in the plots are scatter around the lower left corner (for $y<=40000$) with two accumulations of values spreading along the $x$-axis around $y=100000$ and $y=0$.
The goal of the synthetic data generation method is to produce synthetic data whose \gls{pca} plot is similar to the \gls{pca} plot of the real testing dataset.

\begin{figure}[h]
	\centering
	\begin{subfigure}{0.3\textwidth}
		\centering
		\includegraphics[width=\textwidth]{images/pca/tvae.jpg}
		\caption{TVAE$^{ml}$}
	\end{subfigure}
	\begin{subfigure}{0.3\textwidth}
		\centering
		\includegraphics[width=\textwidth]{images/pca/ctabgan.jpg}
		\caption{CTABGAN$^{ml}$}
	\end{subfigure}
	\begin{subfigure}{0.3\textwidth}
		\centering
		\includegraphics[width=\textwidth]{images/pca/ctabgan+.jpg}
		\caption{CTABGAN+$^{ml}$}
	\end{subfigure}
	\begin{subfigure}{0.3\textwidth}
		\centering
		\includegraphics[width=\textwidth]{images/pca/smote.jpg}
		\caption{SMOTE}
	\end{subfigure}
	\begin{subfigure}{0.3\textwidth}
		\centering
		\includegraphics[width=\textwidth]{images/pca/tab-ddpm.jpg}
		\caption{TabDDPM$^{ml}_q$}
	\end{subfigure}
	\caption{\gls{pca} for Baseline models.}
	\label{fig:pca_base}
\end{figure}

\autoref{fig:pca_base} shows the \gls{pca} plots for the different baseline models. All plots show some similarity with the \gls{pca}-plot from the original dataset,
except for the TVAE model.

\begin{figure}[h]
	\centering
	\begin{subfigure}{0.2\textwidth}
		\centering
		\includegraphics[width=\textwidth]{images/pca/tab-ddpm-bgm.jpg}
		\caption{TabDDPM-BGM$^{ml}_q$}
	\end{subfigure}
	\begin{subfigure}{0.2\textwidth}
		\centering
		\includegraphics[width=\textwidth]{images/pca/tab-ddpm-ft.jpg}
		\caption{TabDDPM-FT$^{ml}_q$}
	\end{subfigure}
	\begin{subfigure}{0.2\textwidth}
		\centering
		\includegraphics[width=\textwidth]{images/pca/tab-ddpm-simTune-minmax.jpg}
		\caption{TabDDPM$^{s}_m$}
	\end{subfigure}
	\begin{subfigure}{0.2\textwidth}
		\centering
		\includegraphics[width=\textwidth]{images/pca/tab-ddpm-simTune.jpg}
		\caption{TabDDPM$^{s}_q$}
	\end{subfigure}
	\caption{Correlation Matrix difference from selected experiment models.}
	\label{fig:pca_diffusion}
\end{figure}

Only the \gls{bgm} tabular processing mechanism seems to be able to produce a realistic \gls{pca}-plot.
Additionally, the Minmax scaling seems to have an effect on the \gls{pca}-plot for the TabDDPM model, however,
this did not affect the TabDDPM-BGM model that made use of Minmax scaling (\autoref{fig_a:pca_TabDDPMBM}).
This \gls{pca}-plot and the remaining model versions plots can be found at \autoref{A:pca}.

\subsection[]{Distribution Plot}
\label{ch:results-Distr}





\section{Analysis}
\label{ch:results-analysis}
%-------------------------------------------------------------------------


TODO: Continue here
\begin{table}[h]
	\centering
	\caption{Comparison of Model A and Model B}
	\label{tab:model-comparison}
	\begin{tabular}{|l|ccc|ccc|}
		\toprule
		\multirow{2}{*}{\textbf{Metrics}} & \multicolumn{3}{c|}{\textbf{TabDDPM}} & \multicolumn{3}{c|}{\textbf{TabDDPM-BGM}}                                                                                      \\ \cline{2-7}
		                                  & \textbf{ML-Efficacy}                  & \textbf{Similarity Score}                 & \textbf{Diff.} & \textbf{ML-Efficacy} & \textbf{Similarity Score} & \textbf{Diff.} \\
		\midrule
		\multicolumn{1}{|l|}{Accurarcy}   & 0.82                                  & 0.73                                      & +0.            & 0.76                 & 0.81                      & +0.            \\
		\multicolumn{1}{|l|}{Metric 2}    & 0.69                                  & 0.64                                      & +0.            & 0.72                 & 0.71                      & +0.            \\
		\multicolumn{1}{|l|}{Metric 3}    & 0.91                                  & 0.88                                      & +0.            & 0.89                 & 0.86                      & +0.            \\
		\bottomrule
	\end{tabular}
\end{table}



The tuning of the hyperparameters had a significant effect on the performance of the different models.


\chapter{Conclusion}
\label{ch:conclusion}


%-------------------------------------------------------------------------
\section{Future Work}
\label{ch:results-futureWork}

% Training by sampling
% Model architecture
% Adversarial loss (anlehnung an pmse score) (adverserial model soll x'_t-1 (NN prediction) und x_t-1 (original) unterscheiden)
% tabular processing infront of preprocessing--> why not after
% tabular processing in training loop einbauen?

%-------------------------------------------------------------------------
\section{Limitations}
\label{ch:results-limitations}
%-------------------------------------------------------------------------



\cleardoublepage

\phantomsection

\addcontentsline{toc}{chapter}{References}

\printbibliography[heading=bibintoc, title={References}]

%\include{kapitel2}
%\include{kapitel3}
%\include{kapitel4}
%\include{kapitel5}
%\include{kapitel6}x
% --- Anhang ---
\backmatter
\cleardoublepage

\blindtext[8]
\cleardoublepage
%\renewcommand\thesection{A.\arabic{section}}
%\renewcommand\thefigure{A-\arabic{figure}}   
%\renewcommand\thetable{A-\arabic{table}}   

\chapter{Appendix} \label{ch:Appendix}


\section*{Visual Results}
\subsection[]{Correlation Difference Matrix}
\label{A:corr_matrix}

\begin{figure}[h]
	\centering
	\begin{subfigure}{0.3\textwidth}
		\includegraphics[width=\textwidth]{images/correlation_difference/ctabgan_simTune.jpg}
		\caption{CTABGAN$^s_q$}

	\end{subfigure}
    \begin{subfigure}{0.3\textwidth}
        \includegraphics[width=\textwidth]{images/correlation_difference/tvae_simTune.jpg}
        \caption{TVAE$^s$}

    \end{subfigure}
	\begin{subfigure}{0.3\textwidth}
		\includegraphics[width=\textwidth]{images/correlation_difference/tab-ddpm-bgm-simTune.jpg}
		\caption{TabDDPM-BGM$^{s}_q$}

	\end{subfigure}
    \begin{subfigure}{0.3\textwidth}
        \includegraphics[width=\textwidth]{images/correlation_difference/tab-ddpm-bgm-simTune-minmax.jpg}
        \caption{TabDDPM-BGM$^{s}_m$}

    \end{subfigure}
	\begin{subfigure}{0.3\textwidth}
		\includegraphics[width=\textwidth]{images/correlation_difference/tab-ddpm-ft-simTune.jpg}
		\caption{TabDDPM-FT$^{s}_q$}

    \end{subfigure}
    \caption{Correlation difference matrix for different model versions}

\end{figure}

\subsection[]{Principle Component Analysis}
\label{A:pca}

\begin{figure}[h]
	\centering
	\begin{subfigure}{0.3\textwidth}
		\includegraphics[width=\textwidth]{images/pca/ctabgan_simTune.jpg}
		\caption{CTABGAN$^s$}
	\end{subfigure}
    \begin{subfigure}{0.3\textwidth}
        \includegraphics[width=\textwidth]{images/pca/tvae_simTune.jpg}
        \caption{TVAE$^s$}
    \end{subfigure}
	\begin{subfigure}{0.3\textwidth}
		\includegraphics[width=\textwidth]{images/pca/tab-ddpm-bgm-simTune.jpg}
		\caption{TabDDPM-BGM$^{s}_q$}
	\end{subfigure}
	\begin{subfigure}{0.3\textwidth}
		\includegraphics[width=\textwidth]{images/pca/tab-ddpm-ft-simTune.jpg}
		\caption{TabDDPM-FT$^{s}_q$}
    \end{subfigure}
	\begin{subfigure}{0.3\textwidth}
		\includegraphics[width=\textwidth]{images/pca/tab-ddpm-bgm-simTune-minmax.jpg}
		\caption{TabDDPM-BGM$^{s}_m$}
		\label{fig_a:pca_TabDDPMBM}
    \end{subfigure}
    \caption{Principle Component Analysisfor different model versions}
    \label{fig_a:pca_diff}
\end{figure}




\cleardoublepage

% VERZEICHNISSE (Abbildungen, Tabellen)
% Literatur 



\cleardoublepage

% ERKLÄRUNG
\chapter*{Eidesstattliche Versicherung}
\thispagestyle{empty}
\addcontentsline{toc}{chapter}{Eidesstattliche Versicherung}

Hiermit versichere ich an Eides statt, dass ich die vorliegende Arbeit im Masterstudiengang IT Management und -Consulting selbstständig verfasst und keine anderen als die angegebenen Hilfsmittel - insbesondere keine im Quellenverzeichnis nicht benannten InternetQuellen - benutzt habe. 
Alle Stellen, die wörtlich oder sinngemäß aus Veröffentlichungen entnommen wurden, sind als solche kenntlich gemacht.
Ich versichere weiterhin, dass ich die Arbeit vorher nicht in einem anderen Prüfungsverfahren eingereicht habe und die eingereichte schriftliche Fassung der elektronischen Abgabe entspricht.


\vspace{2cm} 

\noindent Hamburg, den \uline{~~~~~~~~~~~~~~~~~~~~}~~~~~Unterschrift: \uline{~~~~~~~~~~~~~~~~~~~~~~~~~~~~~~~~~~~~~~~~~~~~~~~~~~} 

    
\end{document}