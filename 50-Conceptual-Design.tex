\chapter{Conceptual Design}
\label{ch:conceptualDesign}

\section{Requirements}
\label{ch:conceptualDesign-requirements}

% Ähnlich wie bei Masterarbeit "Generierung einer synthetischen Datenhistorie aus einem Datenbank Snapshot"
% Basierent auf ISO 2510 
% Nicht Funktional: Functional Suitability, Maintainability, (Compatibility (reproducability))

% Funktional (ergeben sich aus den Research Questions):
% - Comparability & reproducability (codebase must be able to reproduce existing results)
% - "Special Processing": must be able to handle tabular data as explained in chapter xy
% - "Special Processing": can have multiple versions (bayesian, embedding, etc.) 
% - "Special Processing": different versions should be switchable easily
% - "Special Processing": Provide general framework for possible future special processing types
% - Evaluation: must be able to evaluate the results with statistical similarity measures


\section{Existing Code Base}
\label{ch:conceptualDesign-existingCodeBase}

\subsection{Current Implementation}
\label{ch:conceptualDesign-existingCodeBase-currentImplementation}
% Why did I choose to use the existing code base?
% What are the advantages and disadvantages of the existing code base?
% Diagram explaining the current Software architecture

\subsection{Experiment Run}
\label{ch:conceptualDesign-existingCodeBase-experimentRun}
% What is an experiment run? (train, sample, evaluate, hyperparameter tuning, ...)
% Diagrams explaining the current implementation 
% Hyperparameters

\subsection{Reproduction of existing Results}
\label{ch:conceptualDesign-existingCodeBase-reproductionOfExistingResults}
% Reproduce the results of the paper "Diffusion Probabilistic Models for Tabular Data Synthesis"

\section{Proposed Code Extensions}
\label{ch:conceptualDesign-codeExtensions}

\subsection{"Special Processing"}
\label{ch:conceptualDesign-codeExtensions-dataPreprocessing}

% what is meant with "special processing"? --> transforming the data into a different format
% what is necessary to consider when transforming the data? --> Only fit on train data not on test data, etc. 
% why is it necessary? --> to improve the results
% how is it implemented? --> Processing abstract class with different implementations

% Details for each processing type? or later?

\subsection{Evaluation}
\label{ch:conceptualDesign-codeExtensions-evaluation}

% How and where is the statistical similarity measured? 

\subsection{Experiment Run}
\label{ch:conceptualDesign-codeExtensions-experimentRun}

% How did the Experiment Run change with the code extensions?
% Hyperparameters

\section{Experiments}

% What experiments with which changes are planned?
% Baseline 1: no preprocessing + statistical similarity
% Baseline 2: Other Data Generation algorithms (GAN, ...) + statistical similarity 
% Experiment 1.1: preprocessing Bayesian gaussian mixture + statistical similarity
% Experiment 2.1: preprocessing embeddings + statistical similarity
% ...
% Experiment N: Baseline 1 + hyperparameter optimization based on statistical similarity
% if Experiment N > Baseline 1:
%   Experiment 1.2: preprocessing Bayesian gaussian mixture + hyperparameter optimization based on statistical similarity
%   Experiment 2.2: preprocessing embeddings + hyperparameter optimization based on statistical similarity
%   ...