\chapter{Preliminaries}
\label{ch:preliminaries}

%-------------------------------------------------------------------------
\section{Data Synthesis}
\label{ch:preliminaries-dataSynthesis}

\subsection{Synthetic Data}
\label{ch:preliminaries-dataSynthesis-syntheticData}

% Definition of synthetic data
% Types of synthetic data (Audio, Image, Text, Tabular)
% usage of synthetic data
- synthetic data can be use for testing and validation of applications \cite{gilad2021SynthesizingLinkedData} and machine learning models \cite{dahmen2019SynSysSyntheticData}
- allows data sharing \cite{hernandez2022SyntheticDataGeneration} while fulfilling regulatory and privacy constraints \cite{zhao2022CTABGANEnhancingTabular} (complies with GDPR \cite{zhao2022CTABGANEnhancingTabular})
- rebalance datasets with synthetic data if dataset is skewed \cite{zhao2022CTABGANEnhancingTabular}
- access to data still major bottleneck for researches of ml/dl-models \cite{fan2020RelationalDataSynthesisa}
- access is often restricted due to sensitivity of data (medical records) \cite{esteban2017RealvaluedMedicalTimea}
- can be used as additional training data \cite{kim2021OCTGANNeuralODEbased}

- can be used to create test data for software applications because developers might:\cite{whiting2008CreatingRealisticScenariobased}
    - real data may not be available or not visible for certain developers for security reasons \cite{whiting2008CreatingRealisticScenariobased}
    - test data needs to fulfil certain requirements depending on the testing scenario and obtaining the needed characteristics from real data is time-consuming \cite{whiting2008CreatingRealisticScenariobased}


% differences between real and synthetic data
- synthetic data is cheap to generate \cite{leminh2021AirGenGANbasedSynthetica}
- and can be combined with real data for training purposes \cite{leminh2021AirGenGANbasedSynthetica}

% Challenges of synthetic data (privacy, scalability, availability, etc.)

- two competing objectives in generating synthetic data: \cite{little2021GenerativeAdversarialNetworksa}
    1. high data utility \cite{little2021GenerativeAdversarialNetworksa}
    2. low disclosure risk \cite{little2021GenerativeAdversarialNetworksa}


\subsection{Tabular Data}
\label{ch:preliminaries-dataSynthesis-tabularData}

tabular data is one of the most common ways to maintain massive databases \cite{esmaeilpour2022BidiscriminatorGANTabular}.
tabular data is most common form of structured data \cite{hernandez2022SyntheticDataGeneration}

% Definition of tabular data
underlying properties of tabular data: \cite{zhao2022CTABGANEnhancingTabular}
- Single Gaussian variables:
    Single mode Gaussian distributions are very common.
    The distribution of real data is close to a single mode Gaussian distribution

Mixed data type variables:\cite{zhao2022CTABGANEnhancingTabular}
    a variable can be a mix of these two types + missing values. The Mortgage variable from the Loan dataset is a good example of mixed variable
    According to the data description, a loan holder can either have no mortgage (0 value) or a mortgage (any positive value). 
    In appearance, this variable is not a categorical type due to the numeric nature of the data. 
    So all 4 SOTA algorithms treat this variable as continuous type without capturing the special meaning of the value zero. 
    Hence, all 4 algorithms generate a value around 0 instead of exact 0. And the negative values for Mortgage have no/wrong meaning in the real world

Long tail distributions:\cite{zhao2022CTABGANEnhancingTabular}
    real world data can have long tail distributions where most of the occurrences happen near the initial value of the distribution, and rare cases towards the end
    Real data clearly has 99\% of occurrences happening at the start of the range, 
    but the distribution extends until around 25000. 
    In comparison none of the synthetic data generators is able to learn and imitate this behavior.

Skewed multi-mode continuous variables:\cite{zhao2022CTABGANEnhancingTabular}
    The term multimode is extended from Variational Gaussian Mixtures (VGM)
    This is not a typical Gaussian distribution. There is an obvious peak
    with several other lower peaks
    This behavior is difficult to capture for the SOTA data generator




% Types of tabular data (Numerical, Categorical, etc.)

- Continuous \cite{lederrey2022DATGANIntegratingExperta
}

- Categorical 
    - Binary
    - Nominal (no order)
    - Ordinal (order exists)
\cite{lederrey2022DATGANIntegratingExperta
}

- Dates/Timestamps \cite{hernandez2022SyntheticDataGeneration}

- tabular data can be static (independent rows in a table) or dynamic (time series/multivariate time series) \cite{padhi2021TabularTransformersModeling}


% Challenges of tabular data (mixed data types, missing values, class imbalance, etc.)

Mix of categorical and continouse data types \cite{li2021ImprovingGANInverse}
Unbalanced columns \cite{li2021ImprovingGANInverse}
Continouse columns usually follow complex distributions \cite{li2021ImprovingGANInverse}
values in a dataset can be dependent on other variables \cite{lederrey2022DATGANIntegratingExperta} (also across multiple entries --> find example)

% Preprocessing
different Datatypes require different preprocessing \cite{fan2020RelationalDataSynthesisa} meaningfully \cite{lederrey2022DATGANIntegratingExperta}
categorical:
- ordinal encoding -> assign int to each category
- one-hot -> binary vector for each category

numerical:
- min-max normalization
- GMM-based normalization



\subsection{Synthetic Tabular Data Generation}

- Process or data driven methods \cite{goncalves2020GenerationEvaluationSynthetic}
    - Process driven
        - Simulation models \cite{kowalczyk2022TaxonomyUseSynthetic}
    - Data driven
        - Data Augmentation \cite{kowalczyk2022TaxonomyUseSynthetic}
        - statistical distributions \cite{kowalczyk2022TaxonomyUseSynthetic}
        - ML/DL- based Approaches \cite{kowalczyk2022TaxonomyUseSynthetic}

%challanges
interdependencies between variables must be captures \cite{lederrey2022DATGANIntegratingExperta}
danger of overfitting and generelize relationships between columns which only are present in training data but not in unseen data \cite{lederrey2022DATGANIntegratingExperta}



%-------------------------------------------------------------------------
\section{Deep Learning Architectures}
\label{ch:preliminaries-deepLearningArchitectures}

\subsection{Neural Networks}
\label{ch:preliminaries-deepLearningArchitectures-neuralNetworks}

% brief overview of neural networks (perceptron, multilayer perceptron)
% special neural networks 
% --> convolutional neural networks
% --> recurrent neural networks
% --> Residual neural networks
% --> Attention mechanism

\section{Generative Algorithms}
\label{ch:preliminaries-generativeAlgorithms}

\subsection{Autoencoders}
\label{ch:preliminaries-generativeAlgorithms-variationalAutoencoders}

from \cite{kingma2013AutoEncodingVariationalBayes}

Figure 1 from \cite{razghandi2022VariationalAutoencoderGenerativea}

% what are autoencoders
encoder transforms input into latent space which is reconstructed by decoder \cite{razghandi2022VariationalAutoencoderGenerativea}
--> suffer from "lack of regularity in latent space \cite{razghandi2022VariationalAutoencoderGenerativea}

% what are variational autoencoders
use the KL divergence and encode a gaussian distribution in latent space \cite{razghandi2022VariationalAutoencoderGenerativea}
% how do they work
% mathematical formulation
% Advantages / Problems / Challenges


\subsection{Generative Adversarial Networks}
\label{ch:preliminaries-generativeAlgorithms-generativeAdversarialNetworks}

% what are GAN's
\cite{goodfellow2020GenerativeAdversarialNetworks}
- have been used very sucessfully in many different domains \cite{li2022TTSGANTransformerbasedTimeSeries}
 



% how do they work
\cite{zhao2022CTABGANEnhancingTabular}:
- trained via a zero-sum min-max game 
- the discriminator tries to maximize the objective, while the generator tries to minimize it.
- mentor (D) providing feedback to a student (G) on the quality of his work


\cite{li2022TTSGANTransformerbasedTimeSeries}
- 2 NN (gen and dis)
- gen inp: rand vec of specified dimension; gen out: same dimension, as similar to real training data
- dis: binary classifier, distiguish real and generated data
--> play zero sum game against each other
--> trz to each nash equilibrium

% what improvements have been done to GAN's
- gans struggled with generating discrete variables \cite{torfi2020CorGANCorrelationCapturingConvolutionala}
- State of the art Gan approaches only focused on continouse and categorical types, overlooking mixed data types \cite{zhao2022CTABGANEnhancingTabular}

- conditional GAN \cite{mirza2014ConditionalGenerativeAdversarial}


% mathematical formulation
% Advantages / Problems / Challenges (Mode Collapse, etc.)
- remarkable performance generating syntehtic images and time series data \cite{mckeever2020SynthesisingTabularDatasets}
- struggle with mode collapse --> generate same sample \cite{torfi2020CorGANCorrelationCapturingConvolutionala}
- wasserstein loss helps against mode collapse \cite{frogner2015LearningWassersteinLoss} \cite{arjovsky2017WassersteinGenerativeAdversarial} and has been implemented in many gans (e.g. \cite{zhao2022CTABGANEnhancingTabular})
- gradient penalty \cite{gulrajani2017ImprovedTrainingWasserstein}


\subsection{Transformers}
\label{ch:preliminaries-generativeAlgorithms-transformers}
relies on multiple self-attention layers, surpasses other network architectures and shows properties of universal computation engine \cite{li2022TTSGANTransformerbasedTimeSeries}

% What are transformers
% How do they work
% In what context are they used (usually not for data synthesis)
% Advantages / Problems / Challenges


\subsection{Diffusion Probabilistic Models}
\label{ch:preliminaries-generativeAlgorithms-diffusionProbabilisticModels}

% What are diffusion probabilistic models
first paper \cite{sohl-dickstein2015DeepUnsupervisedLearning}
first famouse paper \cite{ho2020DenoisingDiffusionProbabilistic}
improvements> \cite{nichol2021ImprovedDenoisingDiffusion}
follow up: \cite{dhariwal2021DiffusionModelsBeat}

\cite{ho2022ClassifierFreeDiffusionGuidance}

\cite{rombach2022HighResolutionImageSynthesis}

% mathematical formulation
% How do they work
% In what context are they used (usually image synthesis)
% Advantages / Problems / Challenges


% for tabular data
\cite{zheng2022DiffusionModelsMissing} for missing data

\cite{kotelnikov2022TabDDPMModellingTabular} tabddpm
\cite{hoogeboom2021ArgmaxFlowsMultinomial} multinomial diffusion


\subsection{Diffusion Probabilistic Models for Tabular Data}
\label{ch:preliminaries-generativeAlgorithms-diffusionProbabilisticModelsTabularData}

% How can Diffusion Probabilistic Models be used for tabular data
% Challenges of using Diffusion Probabilistic Models for tabular data (mixed data types --> different noising process, etc.)


%-------------------------------------------------------------------------
\section{Evaluation of Synthetic Tabular Data}
\label{ch:preliminaries-evaluationOfSyntheticTabularData}

- there is no universal metric for data synthesis \cite{hernandez2022SyntheticDataGeneration}
- utility and information disclosure metric dimensions \cite{goncalves2020GenerationEvaluationSynthetic}
- structural similarity \cite{elemam2020SevenWaysEvaluate}

\subsection{Statistical Evaluation}
\label{ch:preliminaries-evaluationOfSyntheticTabularData-statisticalEvaluation}

\subsection{Machine Learning Efficiency}
\label{ch:preliminaries-evaluationOfSyntheticTabularData-machineLearningEfficiency}

\subsection{Privacy Evaluation}
\label{ch:preliminaries-evaluationOfSyntheticTabularData-privacyEvaluation}

\subsection{Additional Evaluation Methods}
\label{ch:preliminaries-evaluationOfSyntheticTabularData-otherMetrics}

% Bias and stability
% Domain Expertise

\subsection{Similarity Score}
\label{ch:preliminaries-evaluationOfSyntheticTabularData-similarityScore}
% https://www.researchgate.net/publication/344227988_On_the_Generation_and_Evaluation_of_Tabular_Data_using_GANs

% TOREAD: https://www.researchgate.net/publication/361949372_TabSynDex_A_Universal_Metric_for_Robust_Evaluation_of_Synthetic_Tabular_Data

%-------------------------------------------------------------------------




