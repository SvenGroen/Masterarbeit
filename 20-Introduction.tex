\chapter{Introduction}
\label{ch:introduction}

\section{Problem Statement and Motivation}
\label{ch:intro-problemStatement}

Tabular data is essential in our society due to its ability to organize and present data in a structured manner
that can be easily analyzed and interpreted by humans and machines alike.
This makes it an indispensable tool for decision-making in many fields, including finance, medical research, etc.
In the era of deep learning and big data, the need for accurate and reliable data in high quantities is paramount.
Training data is essential for machine and deep learning \glspl{model} to perform any desired inference.
Gathering and accumulating real-world data is still expensive \cite{borisov2022DeepNeuralNetworks} and restricted through privacy regulations like \gls{gdpr} \glsadd{gdpraccr} \cite{european_commission_regulation_2016}.
In recent years, generative modeling approaches have addressed these issues by creating a synthetic copy of the original tabular dataset.
This synthetic data version should maintain the important statistical properties and correlations while not disclosing any information from the real original dataset \cite{goodfellow2020GenerativeAdversarialNetworks, mottini2018AirlinePassengerName}.
While generative \glspl{model} in the image domain already show a human-like capability to produce highly realistic images \cite{dhariwal2021DiffusionModelsBeat},
tabular data synthesis approaches still offer room for improvements \cite{chundawat2022UniversalMetricRobust}.
One of the reasons for this is the heterogeneous nature of tabular data, which generative \glspl{model} need to reproduce.
Tabular data usually consist of a mixture of numerical and categorical distributions with dependencies, which is difficult to capture and reproduce \cite{borisov2022DeepNeuralNetworks}.
So far, \acrfull{gan}-based approaches have shown state-of-the-art capabilities to create synthetic tabular data, which have also been the best \gls{model} for synthetic image generation.
However, diffusion-based \glspl{model} have recently been able to outperform \glspl{gan} on the task of image synthesis \cite{dhariwal2021DiffusionModelsBeat}.
As a result, researchers have tried to use diffusion \glspl{model} to generate synthetic tabular data with promising results \cite{kotelnikov2022TabDDPMModellingTabular, zheng2022DiffusionModelsMissing}.

\textcite{dhariwal2021DiffusionModelsBeat} argued that \glspl{gan} have shown superior performance against diffusion because researchers have investigated them more deeply.
The authors showed that diffusion \glspl{model} could outperform \glspl{gan} on image synthesis with a few improvements.
A similar observation can be made in the tabular data domain, where the majority of recent publications propose \gls{gan}-based solutions.
First results \cite{kotelnikov2022TabDDPMModellingTabular} indicate that the superior performance of diffusion \glspl{model} over \glspl{gan} extends to the tabular data synthesis domain.
However, applying diffusion \glspl{model} to tabular data is still a novel approach, and further research is required \cite{borisov2022DeepNeuralNetworks}.


\section{Goals and Research Questions}
\label{ch:intro-goals}
This thesis aims to further investigate how diffusion \glspl{model} can be successfully applied to tabular data synthesis.
This work will build upon the work of Kotelnikov, Baranchuk, Rubachev, and Babenko \cite{kotelnikov2022TabDDPMModellingTabular} and will explore how the existing approach called TabDDPM can be further improved by adapting
already existing tabular processing mechanisms from previous tabular data synthesis solutions.
Furthermore, the evaluation and comparison of the diffusion \gls{model} to various other generative modeling techniques performed by \textcite{kotelnikov2022TabDDPMModellingTabular} will be extended through the usage and analysis
of multiple similarity metrics as proposed by \textcite{chundawat2022UniversalMetricRobust}.
This extended evaluation aims to uncover potential unknown modeling properties of diffusion \glspl{model} that may not have been discovered in previous works.

To summarize, this thesis will investigate the following \glspl{rq}:

\begin{description}
	\item[\gls{rq}1:] What are the effects on the generative capability of the tabular data synthesis \gls{model} TabDDPM by adding additional tabular data processing mechanisms to the existing generation pipeline?
	\item[\gls{rq}2:] How does TabDDPM (and its variants, produced to answer \gls{rq}1) compare to other generative \glspl{model} in an extended similarity evaluation, based on the metrics of \textcite{chundawat2022UniversalMetricRobust}?
\end{description}

Answering the above questions contributes to the overall research area by showing possible strengths and weaknesses of diffusion \glspl{model}
as a new approach in the tabular data synthesis domain.

\section{Procedure Overview}
\label{ch:intro-proceeding}
In order to answer the above-stated \glspl{rq}, the following steps have been taken:
Prior to any experiments, a literature search has been performed about different possible approaches to processing tabular data, tabular data synthesis, and how to evaluate synthetic data.
This research led to the selection of tabular processing mechanisms based on predefined criteria (\Autoref{ch:Concept-criteria}).
Using the existing TabDDPM \gls{model} \cite{kotelnikov2022TabDDPMModellingTabular}, the results of the original authors needed to be replicated and validated in the first step.
Afterwards, the software was extended to integrate the selected tabular processing mechanism.
The existing evaluation proposed in \cite{kotelnikov2022TabDDPMModellingTabular} was extended by an additional, more elaborate evaluation \cite{chundawat2022UniversalMetricRobust}.
Lastly, the different \gls{model} versions have been trained, and the results have been compared and analyzed.

\section{Contributions}
\label{ch:intro-contributions}
This thesis provides several contributions to the field of diffusion-based tabular data synthesis.
These contributions expand on the understanding of diffusion \glspl{model}' performance and potential improvements in generating synthetic tabular data.
The main contributions of this thesis are as follows:
\begin{enumerate}
	\item \textbf{Investigation of tabular processing mechanisms:}
	      This thesis investigates the impact of different tabular processing mechanisms on an existing diffusion \gls{model} architecture for data synthesis.
	      By comparing a \acrfull{ft} and a \acrfull{bgm} processing technique with several baseline \glspl{model}, new insights are gained into how the encoded data format affects the generative capabilities of diffusion \glspl{model}.
	\item \textbf{Extension of the existing tabular data generation pipeline:}
	      The current TabDDPM \cite{kotelnikov2022TabDDPMModellingTabular} pipeline is extended to allow for easy implementation of different tabular encoding and decoding strategies.
	      This flexibility enables researchers to experiment with various data processing methods and assess their impact on the diffusion \gls{model}'s performance in generating synthetic data.
	\item \textbf{Comprehensive evaluation:}
	      This thesis introduces a comprehensive evaluation that combines machine learning efficacy,
	      TabSynDex similarity metrics and visualization of dataset characteristics for analysis.
	      This multi-faceted evaluation provides a more holistic view of the generated synthetic tabular data,
	      uncovering strengths and weaknesses from different perspectives.
	\item \textbf{Identification of diffusion \glspl{model}' superiority:}
	      Through extensive experiments and evaluations, this thesis shows that diffusion \glspl{model}, specifically TabDDPM-BGM, outperform other generative \glspl{model}, such as \gls{gan} and \gls{vae} in producing synthetic tabular data.
	\item \textbf{Emphasis on hyperparameter tuning:}
	      This thesis highlights the importance of hyperparameter tuning in diffusion \glspl{model} for tabular data synthesis.
	      It shows that appropriate tuning can significantly enhance the performance of diffusion \glspl{model} across various metrics,
	      making them more versatile and adaptable than other generative \glspl{model}.
\end{enumerate}

\newpage
\section{Outline}
\label{ch:intro-outline}

This thesis starts with essential background information required for a complete understanding of tabular data synthesis in \Autoref{ch:preliminaries}.
This is followed by an outline of the most significant related work in \Autoref{ch:relatedWork}, which highlights relevant \glspl{model} from the literature.
In \Autoref{ch:conceptualDesign}, the requirements for the software are listed, and an explanation of the codebase is presented. 
An expansion of the codebase and a description of the implemented changes follows this. 
The subsequent chapter, \Autoref{ch:methodology}, elucidates how the software was implemented and the architectural changes made.
The experiments' results are presented and analyzed in \Autoref{ch:results}. 
Afterward, possible limitations of the results are presented. 
Finally, \Autoref{ch:conclusion} summarizes the key discoveries of this thesis, and provides an outline of possible future research.