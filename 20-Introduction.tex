\chapter{Introduction}
\label{ch:introduction}

\section{Problem Statement and Motivation} 
\label{ch:intro-problemStatement}

- success of DNNs \cite{borisov2022DeepNeuralNetworks}
- deployment of data-driven application requires solving 3 challenges (inference, data generation & interpretability) \cite{borisov2022DeepNeuralNetworks} (Focus on data generation)
- inference is most crucial but needs preprocessed training data with good quality --> data generation to address challenges (missing values, rebalancing, etc) \cite{borisov2022DeepNeuralNetworks}
- it might be simply impossible to use the actual data due to privacy concerns \cite{borisov2022DeepNeuralNetworks}

% For what do we need data? --> data is used everywhere
% Problems of real data (scarcity, privacy, etc.)
% What is synthetic data in 1 sentence
% What is special about tabular data
% use cases of synthetic data (privacy, scalability, availability, etc.)
% Why to use diffusion for synthetic data generation


% Motivation: Diffusion Probabilistic Models
% GAN's very successful for tabular data synthesis (plateau reached?)
% Diffusion beat GAN's in image synthesis (why not for tabular data?)
% first research shows that diffusion is better than GAN's for tabular data
% several improvements had been done to GAN's but not yet for Diffusion models
% Question arises: Can improvements from GAN's be transferred to Diffusion models?

\section{Goals and Research Questions}
\label{ch:intro-goals}

% Goals: extend the existing code base for special data preprocessing, statistical similarity (and maybe U-Net architecture)
% Goals: analysis on how to improve a Diffusion model for tabular data synthesis

% Research Questions:
% 1. does preprocessing the data into a specific format improve the results?
% 2. how good is diffusion for tabular data synthesis in terms of statistical similarity?
% 3. does a special tabular data synthesis metric improve the results when used for hyperparameter optimization?

% (Wenn Zeit übrig ist):
% 3. does using a U-Net architecture improve the results? (maybe)
% 4. Can the similarity score be incorporated into the loss function? (maybe)


\section{Proceeeding}
\label{ch:intro-proceeeding}

% Proceeeding:
% Literature research on Diffusion models and GAN's
% --> What has been done, what has worked, what has not worked, what can be used for diffusion models
% --> decide which improvements are most promising and will be implemented
% Use existing code base for Diffusion models
% --> implement improvements proposed for GAN's and apply to them Diffusion
% --> train different Diffusion versions
% --> Evaluate the results with additional meassures (statistical meassures)
% Compare the results with the current state of the art
% Analyze the results and draw conclusions (what works, what doesn't work, what can be improved)


\section{Contributions}
\label{ch:intro-contributions}
%Problem Statement: Clearly state the research problem being addressed and explain why it is important. Provide a detailed definition of the problem and explain how it relates to the field of tabular data synthesis.

%Research Objectives: Define the specific research objectives of the thesis and explain how they contribute to the overall research problem.

%Research Methodology: Describe the methodology used to address the research problem, including the techniques and tools used, the data sources used, and the data analysis techniques used.

\subsection{Data Preprocessing}
\label{ch:intro-contributions-dataPreprocessing}

% Data preprocessing is a crucial step in data synthesis
% Use special preprocessing for tabular data
% flexible and extendable preprocessing module inside the current implementation

\subsection{Model Architecture} % maybe 
\label{ch:intro-contributions-modelArchitecture}

% Current Diffusion models are based on simple MLP's
% U-Net architecture is used for image synthesis
% Use U-Net architecture for tabular data synthesis?

\subsection{Extended Evaluation}
\label{ch:intro-contributions-extendedEvaluation}

% Current work focuses on Machine learning efficiency
% Add additional evaluation metrics
% "Similarity Score" used from Master thesis paper that aggregates several metrics


\section{Outline}
\label{ch:intro-outline}
