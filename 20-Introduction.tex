\chapter{Introduction}
\label{ch:introduction}

\section{Problem Statement and Motivation} 
\label{ch:intro-problemStatement}

Tabular data is an essential tool in our society due to its ability to organize and present data in a structured manner that can be easily analyzed and interpreted by humans and machines alike. 
This makes it an indispensable tool for decision-making processes in many fields, including finance, medical research, and many more.
In the era of deep learning and big data, the need for accurate and reliable data in high quantities is paramount.
For machine and deep learning models, training data is essential to perform any kind of desired inference.
Gathering and accumulating real world data is still expensive and restricted through regulations like GDPR.
In recent years, generative modeling approaches address these issues by creating a synthetic copy of the original tabular dataset.
This synthetic data version should maintain the important statistical properties and correlations while not disclosing any information from the real original dataset \cite{goodfellow2020GenerativeAdversarialNetworks, mottini2018AirlinePassengerName}.
While in the image domain, generative models already show human-like capability to produce highly realistic images \cite{dhariwal2021DiffusionModelsBeat},
tabular data synthesis approaches still show room for improvements \cite{chundawat2022UniversalMetricRobust}.
One of the reasons for this, is heterogeneous nature of tabular data, which generative models need to reproduce.
Tabular data usually consist of a mixture of numerical and categorical distributions with dependencies, which is difficult to capture and to reproduce \cite{borisov2022DeepNeuralNetworks}.
So far, \gls{gan} based approaches have shown state-of-the-art capabilities to create synthetic tabular data, which have also been the best model for synthetic image generation.
However, diffusion based models have recently been able to outperform \glspl{gan} on the task of image synthesis \cite{dhariwal2021DiffusionModelsBeat}.
As a result, researchers have tried to use diffusion models to generate synthetic tabular data with promising results \cite{kotelnikov2022TabDDPMModellingTabular, zheng2022DiffusionModelsMissing}.

\cite{dhariwal2021DiffusionModelsBeat} argued, that \glspl{gan} have shown superior performance against diffusion due to the fact, that they have been investigated more deeply by researchers.
The authors were able to show in their work, that with a few improvements, diffusion models were capable of outperforming \glspls{gan} on image synthesis.
A similar observation can be made in the tabular data domain, where the majority of recent publications propose \gls{gan} based solutions.
First results \cite{kotelnikov2022TabDDPMModellingTabular} indicate the superior performance of diffusion models over \glspl{gan} extends to the tabular data synthesis domain.
Nevertheless, applying diffusion models to tabular data is still a novel approach and further research is required \cite{borisov2022DeepNeuralNetworks}.

 
\section{Goals and Research Questions}
\label{ch:intro-goals}
The overall goal of this thesis is to further investigate how diffusion models can be successfully applied to the task of tabular data synthesis.
This work will build upon the work of \cite{kotelnikov2022TabDDPMModellingTabular} and will explore, how the existing approach, called TabDDPM, can be further improved by adapting 
already existing tabular processing mechanism from previous tabular data synthesis solutions.
Furthermore, the evaluation and comparison of the diffusion model to various other generative modeling techniques will be extended through the usage and analysis
of multiple similarity metrics as proposed by \cite{chundawat2022UniversalMetricRobust}.
The goal of this extended evaluation is to uncover possible unknown modeling properties of diffusion models that have not been discovered in previous works.

To summarize, this thesis will investigate the following \glspl{rq}:

\begin{description}
    \item[\gls{rq}1:] What are the effects on the generative capability of the tabular data synthesis model TabDDPM by adding additional tabular data processing mechanisms to the existing generation pipeline.
    \item[\gls{rq}2:] How does TabDDPM (and its variants, produced to answer \gls{rq}) compare to other generative models in terms of several similarity metrics, including \cite{chundawat2022UniversalMetricRobust}.
\end{description}

Answering the above questions contributes to the overall research domain by showing possible strengths and weaknesses of diffusion models
as a new approach in the tabular data synthesis domain. 

\section{Proceeding}
\label{ch:intro-proceeding}
In order to answer the above stated \glspl{rq}, the following steps have been taken.
Prior to any experiments, a literature search has been performed about possible different approaches to process tabular data and tabular data synthesis and how to evaluate it.
From this research, a selection of tabular processing mechanism was made based on predefined criteria (\autoref{ch:Concept-criteria}).
Using the existing TabDDPM model \cite{kotelnikov2022TabDDPMModellingTabular}, the results of the original authors needed to be replicated and validated in a first step.
Afterwards, the software was extended to integrate the selected tabular processing mechanism.
The already existing evaluation proposed in \cite{kotelnikov2022TabDDPMModellingTabular} was extended by an additional, more elaborate evaluation \cite{akim2023TabDDPMModellingTabular}.
Lastly, the different model versions have been trained and the results have been compared and analyzed.

\section{Contributions}
\label{ch:intro-contributions}

\section{Outline}
\label{ch:intro-outline}
