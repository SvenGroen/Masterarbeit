% \phantomsection
% \addcontentsline{toc}{chapter}{Abstract}
\thispagestyle{empty}
\begin{center}
	\textbf{\LARGE Abstract}
\end{center}
The growing demand for data in machine learning and deep learning applications, combined with the difficulties in acquiring and gathering real-world data, has fueled the development of synthetic data generation techniques.
Synthetic data, which is artificially generated but modeled on real data, can address privacy constraints and provide a cost-effective alternative for various use-cases.
With diffusion models, a newly emerged data generation technique, the quality of synthetic images increased and outperformed state-of-the-art \gls{gan}-based solutions. 
Recently, TabDDPM, a generative diffusion model, outperformed \glspl{gan} on tabular data synthesis as well.
Synthesizing tabular data presents unique challenges due to the complexity of the underlying joint distributions between variables and the need to capture intricate relationships among features.
While \gls{gan} based solutions have been heavily explored in the literature, diffusion-based solutions are relatively new and unexplored.


This thesis investigates whether adapting different tabular processing mechanisms from the literature positively affects the diffusion model's generative capability by encoding the tabular data into a different data format that specifically addresses known challenges of tabular data.
By extending the existing tabular data generation pipeline of TabDDPM, various tabular encoding and decoding strategies are implemented.
It focuses on two tabular processing mechanisms that encode the data before training and revert it back after sampling synthetic data: a static embedding technique named \gls{ft} and a mode-specific-normalization encoding technique that makes use of a \gls{bgm}.
In addition to that, this study touches upon the effects of normalization and hyperparameter optimization on diffusion models for tabular data synthesis.
The extended pipeline was evaluated using a CatBoost-based machine learning efficacy evaluation, a similarity metric TabSynDex, covering multiple similarity factors, and an analysis of visualization of the synthetic data.


The results demonstrate the superiority of diffusion models over other generative techniques, such as \Glspl{gan} and \gls{vae}, in generating synthetic tabular data.
The addition of a \gls{bgm}-based processing mechanism improves the TabDDPM model's performance in machine learning scenarios, while the \gls{ft} approach fails to produce meaningful data. 
The importance of hyperparameter tuning and data normalization strategies is highlighted, as well as the need for visual evaluation to accurately assess synthetic data quality. 
The effect of tuning the hyperparameters toward the TabSynDex similarity metric is bigger on diffusion models than on non-diffusion models.
This indicates the diffusion models' flexibility to generate synthetic data to cater to the specific requirements of the intended use case.
Lastly, diffusion models were capable of producing synthetic data that is more non-differentiable from the real data compared to non-diffusion models, which was indicated by a non-zero \gls{pmse} score. 

\cleardoublepage 
