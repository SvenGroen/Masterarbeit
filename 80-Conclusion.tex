\chapter{Conclusion}
\label{ch:conclusion}

\section*{Conclusion}
\label{ch:conclusion}

This thesis investigates the effect of different tabular processing mechanisms on diffusion-based tabular data synthesis.
An already existing tabular data generation pipeline has been extended to implement different tabular encoding and decoding strategies easily.
With this, the existing pipeline of TabDDPM \cite{kotelnikov2022TabDDPMModellingTabular} was extended by two tabular processing mechanisms that encode the tabular data before the training and revert the data back after sampling synthetic data.
The extended pipeline was evaluated using a CatBoost-based machine learning efficacy evaluation, a TabSynDex metric, and visualization of the synthetic data.
The TabSynDex metric comprises five sub-metrics that capture different similarity aspects of the synthetic data.
Several observations could be made in a set of experiments.
Firstly, the results in terms of machine learning efficacy performance of the authors of the TabDDPM pipeline could be reproduced \cite{kotelnikov2022TabDDPMModellingTabular}.
Thanks to the addition of the TabSynDex metrics, it has been shown that TabDDPM superior performance against a set of non-diffusion baseline models also extends to the TabSynDex similarity metrics.

While adding a Feature Tokenization (\gls{ft}) processing mechanism into the TabDDPM generation pipeline transforms the tabular data in such a way,
that the diffusion model was not able to generate any meaningful data, which could be seen in the visual result.
Adopting a more complex encoding/decoding strategy from \cite{zhao2022CTABGANEnhancingTabular}, which uses a Bayesian-Gaussian-Mixture model (\gls{bgm}),
improved TabDDPM's generative capabilities in producing synthetic data that is useful in machine-learning scenarios.
This has shown that changing the tabular data format greatly affects the diffusion model's generative capability.

Furthermore, the importance of hyperparameter tuning for diffusion models was highlighted.
Diffusion models tuned towards the TabSynDex similarity score greatly improved several of the TabSynDex metrics.
In the performed experiments, diffusion models have been the first models to achieve a non-zero \gls{pmse} score, which other current state-of-art models have not been capable of.
Further experiments showed how changing the data normalization strategy can, on the one hand, increase metric results but, on the other hand, worsen visual results.
This insight underlines the importance of a visual evaluation of synthetic data to estimate the quality of the produced synthetic data properly.
From all performed experiments, it can be concluded that diffusion-based tabular data synthesis is superior to the synthesis of other model types, such as \gls{gan} or \gls{vae}.
Diffusion models have not only shown overall better metric and visual results, but they are also more flexible, as a hyperparameter tuning already affects the outcome of the metrics greatly.
Out of all models, TabDDPM-BGM$^{s}$ models showed the best overall performance.
However, potential users should alter the model pipelines design depending on the use case of the synthetic data, with TabDDPM-BGM$^{s}_m$ producing the best correlation score, TabDDPM-BGM$^{ml}_q$ best machine learning efficacy scores and TabDDPM-BGM$^{s}_q$ best basic statistical values and \gls{pmse} results.

